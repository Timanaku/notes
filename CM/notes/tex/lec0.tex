\section{Lecture 0 - 17 Jan 2021 - Revision}
\subsection{The field of complex numbers}
\begin{enumerate}
  \item $\CC$ as a set is in bijection with $\RR\times\RR$,
  \item however $\CC$ is a \emph{field} of characteristic zero, i.e. you can
    keep adding $1$ and it will never get back to $1$ (being in essence
    different from $\ZZ/p\ZZ$ fields for some prime $p$).
  \item Moreover, $\CC$ is complete -- i.e. given a sequence $(z_n)_n\in\CC$
    that is Cauchy, that sequence converges in $\CC$. We used the same in $\RR$. 
  \item  Finally, we say $\CC$ is algebraically closed: any polynomial has
    solutions in $\CC$.
\end{enumerate}

\begin{theorem}[Fundamental Theorem of Algebra]
  Let $f(z)$ be a monic polynomial of degree $n$ with complex coefficients. THe
  equation $f(z)=0$ has precisely $n$ solutions $\{\xi_1,\cdots,
  \xi_n\}\subset\CC$ called roots.
  \label{thm:fta}
\end{theorem}
We will prove this theorem in this course.

\subsection{Complex functions}
From real analysis, we found that given some functions defined in some interval
$[a,b]\to\RR$, the defintions of integration and differentiation involve limits, 
\[f'(x_0)= \lim_{x\to x_0} \frac{f(x)-f(x_0)}{x-x_0}, x_o\in (a,b)\]
\[\int_a^b f(x) dx = \lim_{n\to\infty}\sum_{k=1}^n f(x_k) \frac{b-a}{n}, \quad
x_k \in (a+(k-1)\frac{b-a}{n}, a+k\frac{b-a}{n})\]

In complex numbers, for a function $f:D\to\CC$ defined on an open subset
$D\subset\CC$, we have
\[f'(z_0)= \lim_{z\to z_0} \frac{f(z)-f(z_0)}{z-z_0}, z_o\in D\]
However, in $\RR^2$ we could not take the limit of the quotient of an element of
$\RR^2$, but in $\CC$ this limits does exist, and this requirement is way more
restrictive than for $\RR^2$.

\subsection{Miracles of complex analysis}
What are the distinctive features of complex analysis? There're many things that
are true in complex analysis but not true in real analysis.
\begin{enumerate}
  \item Let $f:D\to\CC$ be differentiable on a small disc $D\subset\CC$, then
    $f$ is analytic on $D$. That is, it can be represented as a Taylor series on
    $D$. In real analysis, a functino being differentiable did not automatically
    guaranteed infinitely-times differentiability.
  \item Complex integration is defined along contours in $\CC$. \emph{Cauchy's
    Theorem}: If $f:D\to\CC$ is analytic on $D$, then the contour integral
    doesn't depend on the choice of contour, but only in the endpoints.
  \item \emph{The residue theorem}: Let $\gamma:[0,1]\to D\subset\CC$ be a closed
    contour. The contour integral of a function $f:D\to\CC$ along $\gamma$
    depends only on the nature of the singularities of $f$ which are enclosed by
    that contour.
  \item \emph{Analytic functions} $f:\CC\to\CC$ are conformal transformations of
    the complex plane -- they preserve angles.
\end{enumerate}
\subsection{Revision of complex numbers}
We revise Level 1 material. Recommended reading is Pristley - Intro to Complex
Analysis, Chapter 1. We cover addition, multiplication, conjugation, Euler's
fomular, de Moivre's theorem, and inverse function. We also cover addition of
complex numbers as translations, and multiplication as scaling plus rotation.
