\section{Lecture 3 - 19 Jan 2021 - Complex power series}
The first example of a holomorphic function is the power series.
\begin{definition}[Power series]
  A complex power series is an infinite sum of the form
  \[S(z) = \sum_{n=0}^{\infty}a_n(z-c)^n\]
  With $a_i,c\in\CC$, where we say that the power series is centered at $c$.
  Given a power series $S(z)$ we say it converges at $z_0\in\CC$ if the limit
  $\lim m\to\infty S_m(z_0)=\sum_{n=0}^{m} a_n(z_0-c)^n$ exists. Otherwise we
  say that $S(z)$ diverges at $z_0\in\CC$.
  \label{def:powerSeries}
\end{definition}
A simple example of a power series is a geometric series.
\begin{definition}
  Consider the power series $S(z)=\sum_{n=0}^{\infty}a_n(z-c)^n$ and the $m$th
  partial sum $S_m(z)$ that converges to $S(z)$. We say that $S(z)$ is
  \emph{absolutely convergent} at $z=z_0$ if $\sum_{n=0}^{\infty}|a_n||z-c|^n$.
  converges.

  We say that $S(z)$ is \emph{uniformly convergent} in an open disk $|z-c|<r$
  centered at $c$ if 
  \[\forall \eps>0 \exists n\in\NN : m>n \implies |S_m(z)-S(z)|<\eps \forall
  z\in(c-r,c+r)\]
\end{definition}
\begin{remark}
  How is uniform convergence different from normal convergence? Observe that $n$
  here depends solely on $\eps$, and not on $z$. This is analogous as for
  functions that are uniformly continuous.
\end{remark}

\begin{theorem}[Radius of convergence]
  Consider a power series $S(z)$ centered at $c$. There exists a positive real
  number $R$, called the \emph{radius of convergence}, s.t. 
  \begin{enumerate}
    \item $S(z)$ is divergent for all $z\not\in (c-R,c+R)$,
    \item For every $r\in[0,R)$, the series $S(z)$ is absolutely and uniformly
      convergent for all $z\in(c-r,c+r)$,
    \item $R=\sup\{r: |a_n|r^n \text{ is a bounded sequence}\}$
  \end{enumerate}
  In the case where $S(z)$ converges for all $z\in\CC$, we say $R=\infty$.
  \label{def:radiusOfConvergence}
\end{theorem}

\begin{proposition}[Ratio Test]
  A complex power series $S(z)=\sum_{n=0}^{\infty} a_n(z-c)^n$ with 
  \[\lim_{n\to\infty} \left|\frac{a_{n+1}}{a_n}\right| = \frac{1}{R}\]
  Has a radius of convergence $R>0$.
  \label{prop:ratioTest}
\end{proposition}

A surprising application is considering the derivative and antiderivative of a
complex power series, $S'(z)$ and $F(z)$, which turns out both still satisfy the
ratio test, with the same ratio of convergence.

\begin{proposition}
  Let $S(z)=\sum_{n=0}^{\infty} a_n(z-c)^n$ be a power series with radius of
  convergence $R>0$. Then $S:D\to\CC:z\mapsto S(z)$ is holomorphic, where
  $D=\{z\in\CC : |z-c|<R\}$ and $S'(z) = \sum_{n=0}^{\infty} a_n
  n(z-c)^{n-1}$.
  \label{prop:powerSeriesIsHolom}
\end{proposition}
\begin{remark}
  One of the main results from complex analysis, which we will see eventually in
  the course, is the fact that every holomorphic series can be written as a
  power series.
\end{remark}
\begin{corollary}
  Every power series $S(z)$ equals its own Taylor series,
  \[S(z) = \sum_{n=0}^{\infty} \frac{S^{(n)}(c)}{n!} (z-c)^n\]
  And posseses antiderviative $F(z)=w+\sum_{n=0}^{\infty}
  \frac{a_n}{n+1}(z-c)^{n+1}$ with radius of convergence $R$ and some constant
  $w\in\CC$.
\end{corollary}
