\section{Lecture 7 - 08 Feb 2022 - Complex integration}
Integration in the complex plane, like in the case for $\RR^2$, will be carried
out along a path.
\begin{definition}
  A $C^1$-curve in an open subset $\Omega\subset\CC$, defined by
  $\gamma:[a,b]\to\CC$, is called a \emph{path} in $\Omega$. Here
  $[a,b]\subset\RR$.
\end{definition}
The basic notion for complex integration is simple: Given a path $\gamma$, the
so-called contour integral of $f:\Omega\to\CC$ with $\gamma([a,b])\subset\Omega$
is given by
\[\int_{\gamma} f(z)dz = \int_{\gamma} (u(x,y)+iv(x,y)) (dx+i dy) =
\int_{\gamma}(u(x,y)dx - v(x,y)dy) + i\int_{\gamma}(v(x,y)dx + u(x,y) dy).\]
Hence we see that the complex contour integral is the complex sum of two real
line integrals. With a more rigorous definition, we can look at the integration
as the limit of the Riemann sum along the path $\gamma$. In particular, we
consider some partition $[a,b]$ as $[a,a_1],[a_1,a_2],\cdots,[a_{n-1},a_n]$,
where $a_n=b$, and then observe that the Riemann sum is 
\[\sum_{r=1}^n f(\gamma(t_r)) (\gamma(a_r) - \gamma(a_{r-1})), a_{r-1}< t_r<
a_r.\]
Note that the Riemann sum only depends on the point $\gamma(a_i)\in\CC$, and on
$f(\gamma(t_i))$, but in particular it does not depend on the parametrisation of
$\gamma$. For this reason we will often just use parametrisation
$\gamma:[0,1]\to\Omega$ without loss of generality, and describe a path in terms
of its image $\gamma([0,1])$ and its orientation (in which order we run through
the points $\gamma(a_0),\cdots, \gamma(a_n)$).
\begin{definition}
  Let $f:\Omega\to\CC$ be continuous and $\gamma:[0,1]\to\Omega$ a path in
  $\Omega$. Then the contour integral of $f$ along $\gamma$ is the limit 
  \[\int_{\gamma} f(z)dz = \lim_{n\to\infty} \sum_{r=1}^{n} f(\gamma(t_r))
  (\gamma(\frac{r}{n}) - \gamma(\frac{r-1}{n})), \frac{r-1}{n}<t_r
  <\frac{r}{n}.\]
  Since $f$ is continuous, the limit does not depend on the choice of $t_r$.
  \label{<+label+>}
\end{definition}

\begin{proposition}
  Consider a path $\gamma:[0,1]\to\Omega$, and a path
  $\overline{\gamma}:[0,1]\to\Omega$ s.t. $\overline{\gamma}(t)=\gamma(1-t)$.
  Then
  \[\int_{\overline{\gamma}}f(z)dz = -\gamma_{\gamma}f(z)dz.\]
  \label{<+label+>}
\end{proposition}
\begin{proposition}
  Let $\gamma:[0,1]\to\Omega$ be a path of finite length -- $\int_0^1
  |\gamma'(t)|dt \leq L <\infty$ (note that $\gamma'(t)$ can be seen as the
  vector tangent to $\gamma$). Consider a function $f:\Omega\to\CC$ bounded on
  $\gamma([0,1])\subset\Omega$ -- $|f(z)|\leq M $ for every
  $z\in\gamma([0,1])$. Then $|\int_{\gamma} f(z) dz|\leq ML$.
  \label{prop:upperBoundIntegral}
\end{proposition}
The above proposition is just saying that the value of the integral will be at
most the area of a rectangle of area $ML$, where $L$ is the length of the base,
and $M$ is the height of the rectangle, where the height is taken to be the
highest value of the function in the path. The proof can be done with riemann
sums.

In applications we usually deal with piecewise smooth curves.
\begin{definition}
  A curve $\gamma:[0,1]\to\Omega$ is called piecewise smooth (piecewise $C^1$)
  if it is differentiable except for finitely many points. We shall refer to
  such a $\gamma$ as path.
\end{definition}

\begin{proposition}
  Let $\gamma:[0,1]\to\CC$ be piecewise $C^1$ and of finite length. Then for any
  continuous $f:\Omega\to\CC$ we have
  \[\int_{\gamma}f(z)dz = \int_{0}^1 f(\gamma(t))\gamma'(t) dt.\]
  \label{<+label+>}
\end{proposition}

\begin{example}
  Let $\gamma:[\alpha,\beta]\to\CC$ be the path 
  \[\gamma(\theta)=c+re^{i\theta},\]
  For some constants $c\in\CC$ and $r>0$. Compute the length of $\gamma$.
\end{example}
\begin{proof}[Solution]
  We have that the lenght of $\gamma$, let us call it $L$, is
  \[L= \int_{\gamma(\alpha)}^{\gamma(\beta)} d\gamma =  \int_{\alpha}^{\beta}
  |\frac{d\gamma}{d\theta}d\theta = \int_{\alpha}^{\beta} rd\theta =
  r(\beta-\alpha).\]
  Note that this path is a circle/segment with center $c$ and radius $r$. Hence
  the length of the curve given above is the formula for length of segment.
\end{proof}
