\section{Lecture 15 - 8 Mar 2022 - Residue Theorem}
This is a method for dealing with poles of finite order. After generalising the
notion of power series to Laurent series for function with isolated
singularities, now we will try to generalise Cauchy's theorem for the the path
encloses a pole.
\begin{definition}
  Let $f:\Omega\setminus\{c\}\to\CC$ be holomorphic and have an isolated
  singularity at $c\in\Omega$. Then, the coefficient $a_{-1}$ in the Laurent
  expansion of $f$ at $c$, with the form
  \[a_{-1}=\frac{1}{2\pi i} \oint_{\gamma} f(z)z,\]
  Is called the residue of $f$ at $c$, denoted $\res(f, c)$. Here $\gamma$ is a
  small circle centered at $c$, traced anticlockwise.
  \label{def:residue}
\end{definition}
\begin{theorem}[Cauchy's Residue Theorem]
  Let $\Omega\subset\CC$ be open and let $f:\Omega\to\CC$ be analytic, and let
  $\gamma$ be a simple closed path in $\Omega$ such that there are finitely many
  points $\{c_1,\cdots, c_k\}$ inside $\gamma$ not belonging to $\Omega$. Then,
  \[\oint_{\gamma} f(z)dz = 2\pi i \left( \res(f,c_1)+\cdots+\res(f,c_k)
  \right).\]
  \label{<+label+>}
\end{theorem}
\begin{proof}
  Let $D_1,\cdots, D_k$ be disjoints open disks inside $\gamma$, centered at
  $c_1,\cdots, c_k$ respectively. It's easy to show that $\gamma$ is homotopic
  to the path made from arcs making up the boundaries of each disk,
  $\gamma(1),\cdots, \gamma(k)$ traced anticlockwise, plus connecting paths
  traced twice in opposite directions (connecting every boundary with the next).
  Think of the homotopy as just squishing $\gamma$ down and as close as possible
  to include only $c_1,\cdots, c_k$. The integrals along the connecting paths
  cancel out, so the result follows by Cauchy's integral formula, applied to
  each singularity,
  
  \[\oint_{\gamma} f(z)dz = \oint_{\gamma(1)}f(z)dz + \oint_{\gamma(2)} f(z)dz
  +\cdots+ \oint_{\gamma(k)}f(z)dz = 2\pi i \left(
  \res(f,c_1)+\cdots+\res(f,c_k) \right).\]
\end{proof}
This generalises Cauchy's theorem (no singularities, so the residue is $0$), and
Cauchy's integral formula (case of $1$ singularity). To compute residues, one
can directly compute Laurent series, or take advantage of some techniques. The
following is a useful result for poles of finite order.

\begin{proposition}
  If $f$ has a pole of degree at most $m$ at $c$, we have 
  \[\res(f,c)= \frac{1}{(m-1)!} \lim_{z\to c} \frac{d^{m-1}}{dz^{m-1}}\left[
  (z-c)^m f(z) \right].\]
  In particular, observe that for a simple pole, 
  \[\res(f,c) = \lim_{z\to c}[(z-c)^mf(z)].\]
  \label{prop:residuePole}
\end{proposition}
\begin{proof}
  Let $g(z)=(z-c)^mf(z)$, so $f$ is analytic at $c$, and so we can write the
  Taylor series at $c$ as $g(z)= a_0 + a_1 (z-c)+\cdots$. Hence, the Laurent
  series of $f$ at $c$ is written as 
  \[f(z)= \frac{1}{(z-m)^m} (a_0 + a_1 (z-c)+\cdots)=
  \frac{a_0}{(z-m)^{m}}+\cdots \frac{a_{m-1}}{(z-m)}+ \cdots\]
  Where we observe that the coeffieicnt of $(z-m)^{-1}$ is precisely the
  residue, i.e. $a_{m-1}=\res(f,c)$. Hence observe that
  $g^{(m-1)}(c)=(m-1)!a_{m-1}$, therefore (since $g$ is analytic)
  \[\res(f,c) = \frac{g^{(m-1)}(c)}{(m-1)!}= \frac{1}{(m-1)!}\lim_{z\to c}
  \frac{d^{m-1}}{dz^{m-1}}\left[ (z-c)^m f(z) \right],\]
  As required.
\end{proof}
In many occassions we en up with a limit quotient, and we can therefore use
L'Hopital's rule to solve it.
\begin{example}
  We can compute the residue of $f(z)=\frac{1}{z^3-1}$. First we identify simple
  poles in $z=1,e^{2\pi i/3}, e^{4\pi i /3}$ by roots of unity. Hence, by the
  above result we find, for each pole $c$,
  \[\res(f,c)= \lim_{z\to c} \frac{(z-c)}{z^3-1}=\lim_{z\to c}
  \frac{1}{3z^2} =\frac{1}{3c^2}. \]
\end{example}
\begin{example}
  Evaluate $\oint_{\gamma} k(z)dz$, where 
  \[k(z)=\frac{1}{z-1}+ \frac{3z+1}{(z-2)^2} + \frac{2}{z-4},\]
  where $\gamma$ is the square with vertices at $\pm 3, \pm 3i$ traced
  anticlockwise. Since there're only 2 singularities inside $\gamma$, naming
  $1,2$, of order $1$ and $2$ respectively, we find (we omit the computation of
  the residues, but it's easily done with the previous result).
  \[\oint_{\gamma}k(z)dz = 2\pi i [\res(f,1)+\res(f,2)] = 2\pi i (4) = 8\pi i.\]
\end{example}

There's a generalisation of the Residue theorem, to any closed path (not only
simple), with essentially the same proof. If $\gamma$ is any closed path in
$\Omega$ and $c$ is not inside the image of $\gamma$, then there's an integer
quantity $w(\gamma, c)$ called the \emph{winding number} defined as 
\[w(\gamma,c)=\frac{1}{2\pi i}\oint_{\gamma} \frac{1}{z-c} dz\]
Representing the number of times $\gamma$ goes around $c$ in an anticlockwise
direction.
\begin{theorem}[Generalised Residue Theorem]
  Let $\Omega\subset\CC$ be open and let $f:\Omega\to\CC$ be holomorphic. Let
  $\gamma$ be a closed path (not necessarily simple). If there are finitely many
  points $c_1,\cdots, c_k$ not in $\Omega$ and enclosed by $\gamma$, then with
  non-zero winding numbers, then
  \[\oint_{\gamma}f(z)dz = 2\pi i [w(\gamma, c_1)\res(f, c_1)+\cdots+ w(\gamma,
  c_k)\res(f, c_k)].\]
  \label{thm:generalResidueThm}
\end{theorem}
