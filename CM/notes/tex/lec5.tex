\section{Lecture 5 - 01 Feb 2022 - Cauchy-Riemann Equations, Harmonic functions}
We will explore more the fundamental difference between complex
differentiability and real differentiability in $\RR^2$. This difference is
embedded in the Cauchy-Riemann equations.

First, identify each complex number $z=x+iy\in\CC$  with its corresponding point
in the complex plan, $z\equiv(x,y)\in\RR^2$. A holomorphic function
$f:\Omega\to\CC$ with $\Omega\subset\CC$ open, is interpreted as a function
$f:\tilde{\Omega}\to\RR^2$, by the aforementioned bijections. We interpret the
domain of $f$ as the $z$-plane, and its codomain as the $w$-plane, where
$w=f(z)=u+vi$. In particular, we interpret $f:(x,y)\mapsto (u(x,y),v(x,y))=(\Re
f(z), \Im f(z))$. 

Recall that for any $z\in\Omega$, the function is differentiable, and by this
definition (as for the real counterpart) we have that the limit
\[\lim_{h\to 0}\frac{f(z+h)-f(z)}{h} =f'(z),\]
must exist for $h\in\CC$, and this must be independent of the argument of $h$.
\todo{I understand this, but why do we put the restriction that $f'$ cannot be a
function of $\arg h$? Why do we require $f'$ to be a function only of $z$? Does
anything weird happen if this restriction is dropped?}
We can specialise to $h=\eps$ and $h=i\eps$, for $\eps\in\RR$ as usual. We find
that 
\[\lim_{\eps\to 0} \frac{f(z+\eps)-f(z)}{\eps} = \lim_{\eps\to 0}\left[ 
    \frac{u(x+\eps, y) - u(x,y)}{\eps} + i \frac{v(x+\eps,y)-v(x,y)}{\eps}
\right]\]
\[=\frac{\partial u}{\partial x} + i\frac{\partial v}{\partial x} =
\frac{\partial f}{\partial x} \]
\[\lim_{\eps\to 0} \frac{f(z+i\eps)-f(z)}{i\eps} = \lim_{\eps\to 0}\left[ 
    \frac{-i(u(x, y+\eps) - u(x,y))}{\eps} +  \frac{v(x,y+\eps)-v(x,y)}{\eps}
\right]\]
\[=-i\frac{\partial u}{\partial y} + \frac{\partial v}{\partial y} =
-i\frac{\partial f}{\partial y}.\]
Henceforth, since we need independence of argument, we have that
\[f'(z)=\frac{\partial u}{\partial x} + i\frac{\partial v}{\partial x} = -i\frac{\partial u}{\partial y} + \frac{\partial v}{\partial y}\]
\[\implies \frac{\partial u}{\partial x} = \frac{\partial v}{\partial y}\land
\frac{\partial v}{\partial x} = -\frac{\partial u}{\partial y}.\]
Note that for differentiablity in $\RR^2$, we don't require this (we can have
  partial differentiation wrt $x$ and wrt $y$, and still works, you get
gradients and other stuff). But since we required the quotient limit to be
independent of the complex number $h$ chosen, these equations follow. In
particular, they state that the image of the square spanned by $\eps, i\eps$ at
$z\in\Omega$ is the square (in the $w$-plane) spanned by $\eps \partial_x f,
\eps \partial_y f$, where $\partial_x f=-i \partial_y f$ which indicates that
$\partial_y f$ is rotated by $\frac{\pi}{2}$ in the $w$-plane.
\begin{remark}
  Note that what the Riemann-Cauchy equations really are saying is that for a
  function to be holomorphic, the aforementioned squares must preserve the shape
  (they may be rotated, translated, but not squished or elongated: the
  perpendicular lines are still perpendicular -- this is a conformal mapping!)
\end{remark}

Hence one can say that holomorphicity is a more demanding criteria than $\RR^2$
differentiablity: we require small squares to be preserved, up to translation
and rotation. However, find find yet another striking result,
\begin{theorem}[The Cauchy-Riemann Equations]
  Let $\Omega\subset\CC$ be open and $f:\Omega\to\CC$ be holomorphic. Define
  $u(x,y)=\Re f(x+iy), v(x,y)=\Im f(x+iy)$. Then, the partial derivatives
  $\partial_x u, \partial_y u, \partial_x v, \partial_y v$ exist on $\Omega\sim
  U\times V$, and satisfy the \emph{Cauchy-Riemann equations}
  \[\frac{\partial u}{\partial x} = \frac{\partial v}{\partial y}\land
  \frac{\partial v}{\partial x} = -\frac{\partial u}{\partial y}.\]

  Conversely, let us define a pair of functions $(u(x,y),v(x,y))$ with
  $u,v:\Omega'\subset\RR^2\to\RR$ whose partial derivatives exist, are
  \emph{continuous}, and obey the Cauchy-Riemann equations. Then, these function
  define a holomorphic function $f:\Omega\to\CC$ as
  \[z\mapsto f(z)=u(\Re z, \Im z) + iv(\Re z, \Im z),\]
  \label{thm:cauchyRiemannEquations}
\end{theorem}
\begin{proof}
  TODO. We can find the lecture notes the converse proof. The first half is in
  fact proven by the introduction to this lecture section.
\end{proof}<++>
\begin{exercise}
  Consider the functions $u=e^x\cos y, v=e^x\sin y$.
\end{exercise}
\begin{proof}[Solution]
  TODO. It's quite fun and rather short. The fun stuff is that, after verifying
  they satisfy Cauchy-Riemann, and that they are $C^1$, it gives that the
  function $f$ they give rise by Theorem \ref{thm:cauchyRiemannEquations} is
  infact just $f(z)=e^z$, and that the requirement $f'(z)=\partial_x
  f(x+iy)=f(z)$ is satisfied.
\end{proof}<++>
