\section{Lecture 2 - 18 Jan 2021 - Intro to Complex Differentiation}
In order to define the derivative of a complex function, we need the notion of
open intervals on $\CC$.

\begin{definition}
  Let $\Omega\subseteq\CC$ be a set. We say $\Omega$ is open iff for any
  $z_0\in\Omega$ there exists $\eps>0$ s.t. 
  \[\{z\in\CC : |z-z_0| < \eps\} \subset \Omega\]
  That is, for any point $z_0\in\Omega$ there exists a small disc
  of radius $\eps$ centered at $z_0$ that is contained in $\Omega$.
  \label{def:openSet}
\end{definition}
Next, we investigate the notion of limits of complex funcitons. 

\begin{definition}
  We write $\lim_{z\to z_0} f(z)= \omega$ iff
  \[\forall\eps>0 \exists \delta_{\eps}>0 : |z-z_0|<\delta_{\eps}\implies
  |f(z)-\omega| < \eps\]
  \label{def:limitComplexFun}
\end{definition}
Note that then, for a limit $\lim_{z\to z_0} f(z)=\omega$, we have that for any
sequence $(z_n)_n$ with $\lim_{n\to\infty} z_n = z_0$, then
$\lim_{n\to\infty}f(z_n)=\omega$.

\begin{definition}
  Let $\Omega\subset\CC$ be an open set, and $f:\Omega\to\CC$ a function. Then
  we say that $f$ is differentiable at $a\in\Omega$ iff 
  \[\lim_{z\to a} \frac{f(z)-f(a)}{z-a} = \omega \in\CC.\]
  Then $\omega$ is called the derivative of $f$ at $a\in\Omega$ and denoted
  $f'(a)$ or $\frac{df}{dz}(a)$.
  \label{def:derivativeComplex}
\end{definition}
The important difference wrt the real analogoous operation is that, in this
case, the limit can take many different direction. However, a requirement here
is that regardless of the direction taken, the limit is the same.
QUESTION: Why would not this hold in $\RR^2$. ANSWER: I think it is because it
is needed for the difference quotient limit to be defined. Without it, we
wouldn't be able to define it. The next examples may clear the doubts.

\begin{example}
  We now look at an example to show the difference betwene a well-behaved
  function in, say, $\RR^2$ and how $\CC$-differentiation needs something more
  than just the normal Cartesian notion of well-behavedness. Take the function
  $f:\CC\to\CC:z\mapsto|z|^2$, which viewing it in the real plane, it would be
  $f_R:\RR^2\to\RR:x\to |x|^2$, describing a parabola. This function $f_R$ is
  nice and differentiable. However, consider the differentiability of $f$, and
  fix $z\in\CC$. Let $\eps>0$ and $\theta\in[0,2\pi)$. The complex difference
  quotient yields
  \[\frac{f(z+\eps e^{i\theta}) - f(x)}{\eps e^{i\theta}} =
  \frac{z\eps e^{-i\theta} + \overline{z}\eps e^{i\theta }+ \eps^2}{\eps
  e^{i\theta}} = ze^{-2i\theta} + \overline{z} + \eps e^{-i\theta}\]
  And by taking the limit $\eps\to 0$ it yields $\omega =
  ze^{-2i\theta}+\overline{z}\in\CC$, which depends on $\theta$, hence it does
  not fulfill the differentiability -- since the limit will not be the same
  regardless of $\theta$. The only point where it is independent is when $z=0$.
  Hence $f$ is only differentiably at $f$.
\end{example}
QUESITON: Is there anything interesting happening if we drop the requirement of
direction-independence?
\begin{exercise}
  Consider the function $f:\CC\to\CC: z\mapsto\overline{z}$. Show that $f$ is
  not differentiable anywhere.
\end{exercise}
\begin{proof}[Solution]
  TODO
\end{proof}

\begin{lemma}[Holomorphic function]
  Consider the function $f:\Omega\to\CC$ for some open set $\Omega\subset\CC$.
  If $f$ is differentiable at every point $\omega\in\Omega$, we say that it is
  \emph{holomorphic}.
  \label{def:holomorphicFun}
\end{lemma}
\begin{proof}
  TODO
\end{proof}

\begin{lemma}[Rules of differentiation]
  Consider the function $f,g:\Omega\to\CC$ to be differentiable at
  $\omega\in\Omega$ with $\Omega\subset\CC$ being open. We have 
  \[(f+g)'(\omega) = f'(\omega)+g'(\omega)\]
  \[(fg)'(\omega) = f'(\omega)g(\omega) + g'(\omega)f(\omega)\]
  And for $\Gamma=\{\omega\in\Omega : f(\omega)\neq 0\}$, we have that for $s\in
  \Gamma$,
  \[\left( \frac{1}{f} \right)'(s) = \frac{-f'(s)}{f(s)^2}\]
  \label{lem:diffRules}
\end{lemma}
\begin{proof}
  TODO
\end{proof}

\begin{lemma}[Chain Rule]
  Let $\Omega_1,\Omega_2\subset\CC$ be open and let $f:\Omega_1\to\CC,
  g:\Omega_2\to\CC$ be holomorphic. Then $g\circ f:\Omega_1\to\CC$ is also
  holomorphic. Moreover, $(g(f(z))'= g'(f(z))f'(z)$ for all $z\in\Omega_1$.
  \label{lem:chainrule}
\end{lemma}
\begin{proof}
  TODO
\end{proof}
