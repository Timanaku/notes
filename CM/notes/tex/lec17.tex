\section{Lecture 17 - 15 Mar 2022 - Conformal Mappings}
In this section we explore the geometric aspects of holomorphic function theory.
We already saw that $f$ being holomorphic is equivalent to $f$ satisfying the
Cauchy-Riemann equatinos, which implies that $f$ preserves $90$ degree angles in
the mapping form the $z$-plane to the $w$-plane.

Let $f$ be holomorphic. We will show that if $f'\neq 0$ on its domain, then $f$
preserves the angle between two curves $\gamma_1, \gamma_2$ (the angle beteween
the tangents of each curve at the intersection). Such maps are called conformal. 
\begin{definition}
  Let $\Omega\in\CC$ be open and $\gamma_1,\gamma_2:[0,1]\to\Omega$ two paths
  which intersect at some point $z=\gamma_1(t_1)=\gamma_2(t_2)$ with $0\leq
  t_1,t_2\leq 1$. If $\gamma_1'(t_1), \gamma_2(t_2)\neq 0$, then $\theta=
  \arctan \left( \frac{\gamma_1'(t_1)}{\gamma_2'(t_2)} \right)$ is the angle
  between both tangent vectors and we say that a holomorphic function
  $f:\Omega\to\CC$ is comformal at $z$ if the tangent vectors of $f\circ
  \gamma_1, f\circ \gamma_2$ at $w=f(z)$ also have the angle $\theta$ between
  them.
  \label{<+label+>}
\end{definition}

\begin{lemma}
  If $f:\gamma\to\CC$ is differentiable at a point $z\in\Omega$ and $f'(z)\neq
  0$ then $f$ is conformal at $z$.
  \label{<+label+>}
\end{lemma}
\begin{proof}
  By using the above definition for the angle,
  \[\frac{(f\circ \gamma_1)'(t_1)}{(f\circ \gamma_2)'(t_2)} =
  \frac{f'(\gamma_1(t_1))\gamma_1'(t_1)}{f'(\gamma_2(t_2))\gamma_2'(t_2)}=
  \frac{\gamma_1(t_1)}{\gamma_2(t_2)},\]
  Which is the definition of the angle between the two tangent vectors in the
  $z-plane$. Hence the map is conformal (preserves angle).
\end{proof}
We now focus on open sets $U\subset\CC$ which are path connected.
\begin{definition}
  Let $U,V\subset\CC$ be open sets, which are path connected. A holomorphic
  function $f:U\to V$ that is also bijective is called biholomorphic, if its
  inverse $f^{-1}:V\to U$ is also holomorphic.
  \label{def:biholomorphicFun}
\end{definition}
\begin{proposition}
  Every biholomorphic map $f:U\to V$ is conformal.
  \label{<+label+>}
\end{proposition}
\begin{proof}
  By the chain rule,
  \[\frac{d}{dz}f^{-1}\circ f (z) = (f^{-1}(f(z)))' f'(z),\]
  And by the definition we have $f^{-1}\circ f(z) = z \forall z\in U$ (since
  we're taking $f$ to be bijective), hence
  \[\frac{d}{dz}f^{-1}\circ f (z) = \frac{d}{dz} z =1.\]
  Hence we see that if $f'(z)=0$, we get a contradictino. Hence we must have
  that $f'(z)\neq 0 \forall z\in U$. 
\end{proof}
\begin{definition}
  We call two, open path connected sets $U,V\subset\CC$ conformally equivalent
  if there exists a biholomorphic map $f:U\to V$.
  \label{def:conformalEquivalence}
\end{definition}
\begin{proposition}
  Conformal equivalence defines an equivalence relation on the set of open
  subset of $\CC$.
  \label{prop:conformalEqRel}
\end{proposition}
\begin{proof}
  Reflection is given by the identity map (which is biholomorphic). Symmetry is
  given by the fact that a biholomorphic map's inverse is also biholomorphic.
  And transitivity comes from composition of biholomorphicity.
\end{proof}
