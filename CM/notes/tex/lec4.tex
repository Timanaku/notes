\section{Lecture 4 - 25 Jan 2022 - Examples of complex functions}
The primal example of important complex function is the function
\[\exp(z)= \sum_{i=0}^{\infty}\frac{z^n}{n!}.\]
Which can be easily checked that has infinite radius of convergence with the
ratio test.

\begin{lemma}[Properties of the exponential function]
  The following hold,
  \begin{enumerate}
    \item $\frac{d}{dz}\exp(z)=\exp(z)$,
    \item $\exp(z)\exp(w)=\exp(z+w)$, and $\exp(-z)=\frac{1}{\exp(z)}$ for all
      $z\in\CC$,
    \item Euler's formula - $\exp(iz)=\cos(z)+i\sin(z)$ for all $z\in\CC$.
      Recall that $\cos(z)= \sum_{n=0}^{\infty}\frac{(-1)^n}{(2n)!}z^{2n}$ and
      $\sin(z)= \sum_{n=0}^{\infty}\frac{(-1)^n}{(2n+1)!}z^{2n+1}$.
  \end{enumerate}
  \label{lem:expProperties}
\end{lemma}
\begin{proof}
  \emph{(1)} From the definition of $\exp(z)$,
  \[\frac{d}{dz}\exp(z)=\frac{d}{dz} \left[ 1 +
    \sum_{n=1}^{\infty}\frac{z^{n}}{n!}\right]= \sum_{n=1}^{\infty}
  \frac{z^{n-1}}{(n-1)!} = \sum_{i=0}^{\infty}\frac{z^n}{n!} = \exp(z).\]

  \emph{(2)} Given the above statement, we have
  $\frac{d}{dz}\exp(z+w)\exp(-z)=\exp(z+w)\exp(-z) - \exp(z+w)\exp(-z)=0$, by
  using the product rule. Since $w$ is some arbitrary constant, we set
  $\exp(z+w)\exp(-z)= \exp(w)$, and the result follows. The next result follows
  by setting $w=-z$.

  \emph{(3)} We have that 
  \[\exp(iz)= \sum_{n=0}^{\infty} \frac{(iz)^n}{n!} = \sum_{n=0}^{\infty}
  \frac{(i)^{2n}z^{2n}}{(2n)!} + \sum_{n=0}^{\infty}
  \frac{(i)^{2n+1}z^{2n+1}}{(2n+1)!},\]
  By splitting even and odd elements. Note that $(i)^{2n}=(-1)^{n}$, and
  $(i)^{2n+1}=(-1)^{n}i$, hence the above rearranges,
  \[\exp(iz) = \sum_{n=0}^{\infty}
  \frac{(-1)^{n}z^{2n}}{(2n)!} + \sum_{n=0}^{\infty}
  \frac{(-1)^{n}iz^{2n+1}}{(2n+1)!} = \cos(z)+i\sin(z),\]
  As required.
\end{proof}

We can look at hyperbolic trigonometric functions as a case application of the
exponential function, but while seeing these definitions, we notice something
odd when trying to look for the respective inverse functions. In particular, we
observe that for complex variables, the exponential funciton is not injective,
\[\exp(z)= \exp(z)\exp(2\pi i).\]
Hence, when using inverse functions, such as complex log or complex roots, we
must define an open set in which these functions are bijective.

\begin{theorem}[Inverse Function Theorem]
  Let $\Omega\subset\CC$ be open and $f:\Omega\to\CC$ be holomorphic. Suppose
  $z\in\Omega$ with $f'(z)\neq 0$. Then there exists $U\subset\Omega$ and
  $V\subset\CC$ containing $z, w=f(z)$, respectively, such that $f:U\to V$ is
  bijective and $f^{-1}:V\to U$ is holomorphic with 
  \[\frac{d}{dz}f^{-1}(w) =\frac{1}{f'(z)}.\]
  \label{thm:inverseFunction}
\end{theorem}
\begin{proof}
  Omitted, but note that by the chain rule, one finds the formula by
  differentiating wrt $z$, $f^{-1}(f(z))=z$.
\end{proof}
Since $e^{z}=e^x(\cos(y)+ i\sin(y)) \neq 0$ for all $z\in\CC$, the inverse
function theorem tells us that for some neighbourhood, i.e. for some open set
containing $z$, the inverse is well defined. Consider the open set
$U=\{z=x+iy|x,y\in\RR, -\pi<x<\pi\}$, and observe that the function
$\exp:U\to\CC\setminus\RR^-$ where $\RR^-=\{x\in\RR : x\leq 0\}$, is bijective.
in particular, observe that $e^{i\pi}=e^{-i\pi}$, hence to have a bijective
exponential function, the domain $U$ must not include $\pi,-\pi$. The inverse
function $\log:\CC\setminus\RR^-\to U$ is called the \emph{principal branch of
the complex logarithm}. Other branches of the complex logarithm exist (in fact,
there're infinitely many!), and one can define a bijective exponential function
with proper open domains (e.g. where $y\in (0,2\pi)$). In general, one refers to
all solutions $w$ of the equation $e^w=z$ as the complex logarithm of $z$.
\begin{example}
  Find all the complex logarithms of $z=-1+i\sqrt{3}$.
\end{example}
\begin{proof}[Solution]
  We have $\log z=\log re^{i\theta} = \log r + i\theta$. Note that, since
  $z=x+iy=re^{i\theta}$,
  \[\arg z=\theta = 
    \begin{cases}
      \tan^{-1}(y/x)+2k\pi, & x>0,\\
      \tan^{-1}(y/x) + (2k+1)\pi & x<0,
    \end{cases}
  \]
  Where $k\in \ZZ$. Note that we need to take into account the two cases where
  $x>0,x<0$ because the tangent function is only defined on $(-\pi/2,\pi/2)$. In
  our case, $r=2$ and $\theta=-\frac{1}{3}\pi + (2k+1)\pi$.  Hence,
  \[\log z = \log r + i\theta = \log 2 + i(-\frac{1}{3}\pi + (2k+1)\pi).\]
\end{proof}

\begin{definition}
  Let $z,w\in\CC$, then we define $z^w:= e^{w\log z}$, where $\log z$ is the
  complex logarithm of $z$
\end{definition}
\begin{lemma}
  Choose the principal branch of the complex logarithm. Then the function
  $f:\CC\setminus\RR^- \to \CC:z\to z^w$ is a well-defined holomorphic
  function for all $w\in\CC$. Moreover, $z^uz^v=z^{u+v}$ for all $u,v\in\CC$ and
  $z\in\CC\setminus\RR^-$.
\end{lemma}

\begin{exercise}
  Find $\arctan z$.
\end{exercise}
\begin{proof}
  Note that we're trying to find $w$ s.t. $\tan w =z=\frac{\sin w}{\cos w} =
  \frac{e^{iw}-e^{-iw}}{i(e^{iw}+e^{-iw})}$. Hence it follows that 
  \[iz(e^{2iw}+1)=e^{2iw}-1 \iff e^{2iw}(zi-1)+(zi+1)=0\]
  \[\iff e^{2wi} = -\frac{zi+1}{zi-1}= -\frac{-z+i}{-z-i}=\frac{i-z}{i+z}\]
  \[\therefore w= \frac{1}{2i}\log\left( \frac{i-z}{i+z} \right)\]
\end{proof}
