\section{Lecture 10 - 24 Feb 2022 - Cauchy's integral formulae}
We will now see some of the benefits and most astonishing results that come from
Cauchy's Theorem. The first result, called Cauchy's Integral Formula, will show
that the values of a holomorphic function $f$ inside some closed path $\gamma$
which is null-homotopic are fixed by the values of $f$ on $\gamma$: The value at
one point is entirely determined by what's going in its surroundings.
\begin{theorem}[Cauchy's Integral Formula]
  Let $\Omega\subset\CC$ be open and $f:\Omega\to\CC$ be holomorphic. Suppose
  $\gamma:[0,1]\to\Omega$ is a simple closed path traced anticlockwise. Then for
  any $z\in\Omega$ which lies inside $\gamma$ we have
  \[f(z)=\frac{1}{2\pi i} \oint_{\gamma} \frac{f(\zeta)}{\zeta-z} d\zeta.\]
  \label{thm:CauchyIntegral}
\end{theorem}
\begin{proof}
  Let $I=\oint_{\gamma}\frac{f(\zeta)}{\zeta-z}d\zeta$. We claim $I=2\pi i
  f(z)$. To do this, let us define $\delta$, a circle with center $z$ and small
  radius $r>0$, traced anticlockwise. For $r$ sufficiently small, $\gamma$ is
  homotopic to $\delta$ in $\Omega\setminus\{z\}$. Note that the integrand of
  $I$ is differentiable on $\Omega\setminus\{z\}$, hence 
  \[I=\oint_{\gamma}\frac{f(\zeta)}{\zeta-z}d\zeta=\oint_{\delta}\frac{f(\zeta)}{\zeta-z}d\zeta,\]
  By Theorem \ref{thm:homotopyInvariance}. Consider the following integral, by
  parametrising $\delta$ as $\gamma=z+re^{it}$ with $t\in[0,2\pi]$,
  \[\oint_{\delta}\frac{f(z)}{\zeta - z} d\zeta =
  f(z)\int_0^{2\pi}\frac{1}{re^{it}}ire^{it}dt=2\pi i f(z).\]
  Hence observe that 
  \[|I-2\pi i f(z)| = |\oint_{\delta}\frac{f(\zeta)-f(z)}{\zeta-z} d\zeta| \leq
  2\pi r |f'(z)+1|, \]
  By using the method of upper-bound for integrals in Proposition
  \ref{prop:upperBoundIntegral}. Hence since $r$ can be arbitrarily small, it
  follows that $I=2\pi i f(z)$, as required.
\end{proof}
\begin{remark}
  Note that in this theorem we don't require any constraints on $\gamma$. It
  just needs to be closed, simple, and enclose the point $z\in\CC$.
\end{remark}
\begin{theorem}[Cauchy's integral formula for derivatives]
  Let $\gamma$ be a simple closed path in $\CC$ traced anticlockwise, let
  $\Omega$ be open and containing $\gamma$ and all points inside, and let
  $f:\Omega\to\CC$ be differentiable. Then $f$ can be differentiated infinitely
  often at any point $z$ inside $\gamma$ and the derivatives for $n\geq 0$ are
  given by
  \[f^{(n)}(z) = \frac{n!}{2\pi i} \oint_{\gamma}
  \frac{f(\zeta)}{(\zeta-z)^{n+1}}d\zeta.\]
  \label{<+label+>}
\end{theorem}

Note that for every point $z$ in an open set $\Omega$, there exists a disk in
$\Omega$ centered at $z$, by defintion of open set. By considering a circle in
this disk with center $z$, we can prove the above result simply by
differentiating the integrand wrt $z$ inductively. Moreover, one also gets the
following result,
\begin{theorem}
  Let $\Omega$ be open and $f:\Omega\to\CC$ be differentiable. Then $f$ be
  differentiated infinitely often.
  \label{thm:complexDiffInfinity}
\end{theorem}

\subsection{Applications}
Recall that a function is \emph{analytic} if it can be locally (on open sets) be
represented as a power series. We already saw earlier that power series are
holomorphic. The converse, that every holomorphic function is analytic, is also
true to an extent by the following theorem,
\begin{theorem}
  Let $\Omega\subset\CC$ be open and $f:\Omega\to\CC$ holomorphic. Then for each
  $c\in\Omega$ there exists an open disk $D_c\subset\Omega$ s.t. 
  \[f(z)=\sum_{n=0}^{\infty} \frac{f^{(n)}(c)}{n!}(z-c)^n,\]
  for all $z\in D_c$. That is, $f$ can locally be represented by its Taylor
  series.
  \label{thm:analyticHolomorphic}
\end{theorem}
\begin{proof}
  Since $\Omega$ is open we can find a disk $D_C\subset\Omega$ centered at $c$
  for every $c\in\CC$. Let $\gamma$ be a closed circle with center $c$
  containing $z\in D_c$. Then by Cauchy's integral formula we have
  \[f^{(n)}(c) = \frac{n!}{2\pi i} \oint_{\gamma} \frac{f(\zeta)}{\zeta-c}
  d\zeta\]
  \[\implies \sum_{n=0}^{\infty} \frac{f^{(n)}(c)}{n!}(z-c)^n=
  \sum_{n=0}^{\infty} \frac{1}{n!} \frac{n!}{2\pi i} \oint_{\gamma}
  \frac{f(\zeta)}{\zeta-c} d\zeta  (z-c)^n.\] 
  \[\implies \cdots = \frac{1}{2\pi i} \oint_{\gamma} \sum_{n=0}^{\infty}
  \frac{f(\zeta) (z-c)^n}{(\zeta-c)^{n+1}}d\zeta.\]
  Next, since we required $z$ to be inside $\gamma$, it follows that 
  \[\left| \frac{z-c}{\zeta -c }\right| < 1\]
  for every $\zeta$ \emph{on} $\gamma$, so the integrand is just a convergent
  geometric series. We can find the sum by treating it as a standard geometric
  series, with sum $S_{\infty} =  a/(1-r)$, 
  \[\sum_{n=0}^{\infty} \frac{f(\zeta) (z-c)^n}{(\zeta-c)^{n+1}} =
  \frac{\left[ \frac{f(\zeta)}{\zeta-c}\right]}{\left[ 1-\frac{z-c}{\zeta-c}
  \right]} = \frac{f(\zeta) }{\zeta-z}. \]
  Hence the integral we're interested in reduces to 
  \[\sum_{n=0}^{\infty} \frac{f^{(n)}(c)}{n!}(z-c)^n = \cdots = \frac{1}{2\pi i}
  \oint_{\gamma} \frac{f(\zeta)}{\zeta-z}dz, \]
  Which is precisely $f(z)$ by Cauchy's integral formula.

\end{proof}
\begin{remark}
  This is really special about holomorphic functions. We saw in real analysis
  that there exists functions whose taylor series exist and converge, but do not
  converge against the original function. However, in holomorphic functions, we
  have that its holomorphic function will not only converge, but also converge
  against the original function.
  \label{<+label+>}
\end{remark}
