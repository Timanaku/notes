\section{Lecture 16 - 10 Mar 2022 - Residue Theorem: Applications}
The residue theorem is one of the most important results that are applicable to
problems outside of pure maths. Consider the computaiton of improper real
integrals, e.g. in transforms. The following two integrals (Fourier and Mellin
transforms) are hard to evaluate by guessing,
\[\int_{-\infty}^{\infty} \frac{e^{ikx}}{1+x^2} dx = \pi e^{-|k|}, k\in\RR\]
\[\int_{0}^{\infty} \frac{x^{\alpha-1}}{1+x}dx =\frac{\pi}{\sin \pi\alpha},
\Re\alpha\in (0,1).\]
In order to approach this problem we can interpret the integral as a section of
a complex closed integral (the path of which has a section going over the real
line interval). We then can apply the residue theorem, and by showing that the
section of the path that is not on the real line vanishes as the length of
the real interval goes to infinity, we obtain the value of the desired integral.
The vanishing of the non-real part integral is non-trivial, and in fact is
simply a condition on the integrand. 

Let $f$ be a function such that $z^2 f(z)$ is bounded for large values of $|z|$
(for large values condition is important, since that give us room to play with
functions that may not be bounded for small values of $|z|$). The integral
\[\int_{-\infty}^{\infty} f(z) dz\]
Is then convergent. By comparison, consider the integral, 
\[\int_{-\infty}^{\infty} \frac{1}{1+x^2}dx =
\lim_{R\to\infty}[\tan^{-1}x]_{-R}^{R} = \pi.\]
In fact, 
\[\int_{-\infty}^{\infty} f(z)dz =\lim_{R\to\infty}\int_{-R}^{R} f(z)dz\]
And this limit can be computed (with the proper constraints on $f$) by
considering complex closed line integrals around boundaries of half disks, 
\[D=\{z\in\CC : |z|\leq R, \Im z \geq 0\}.\]
The boundedness condition of $f$ implies that $zf(z)\to 0$ as $|z|\to \infty$.
From this follows that integrals along semicircular parts of the boundary of $D$
defined above will vanish. Hence it follows that in the limit of large radius of
disk $D$, the improper integral will equal the closed contour integral, which
can be computed simply by using the residue theorem. Look at the examples on the
course notes, they're quite instructive. 

