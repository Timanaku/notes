\section{Week 1 - 10 Jan 2022}
For the first part of the course, we will focus on particle mechanics: the
way discrete dimensionless mass points move and interact with each other. The
next part of the course, will be on the mechanics of rigid bodies: continuous,
dimension-ful mass distributions, with the characteristic that the distance
between any two points in the body do not change under any force -- the way it
moves, wrt the center of mass, is similar to the particle mechanics, with the
added flavour of the rotation about this center of mass. Finally, we will look
at deformable bodies (a solid body for which the aforementioned distance can
change), like beams.

\subsection{Review}
\begin{definition}[Point mass]
  Let $P$ represent some point in space with position vector $x\in\RR^3$ given a
  reference frame $O$. Such a point with a mass $m>0$ is called a \emph{point
  mass}.
  \label{def:pointMass}
\end{definition}

If such a point mass moves during a time interval $[0,T]$, we call it a
trajectory or path. Such a path is described by a time-parametric function,
$x:[0,T]\to\RR^n$. 

\begin{definition}
  The velocity of a point mass $P$ is
  $\vec{v}=\dot{\vec{x}}=\frac{d \vec{x}}{dt}$.
  \label{def:velocity}
\end{definition}
\begin{remark}
  If $|\vec{v}|\neq 0$, we have that $\vec{v}$ is tanger to the point mass path.
\end{remark}
\begin{definition}
  THe speed of $P$ is $v=|\vec{v}|=|\dot{\vec{x}}$.
\end{definition}
\begin{definition}
  The acceleration of $P$ is $\vec{a}=\dot{\vec{v}}=\ddot{\vec{x}}$.
\end{definition}
\begin{remark}
  Note that the rate of change of speed $\dot{v} \neq |\dot{\vec{v}}|$. In other
  words, the magnitude operator and the time-derivative operator do not commute.
  However, $v\cdot v= |\vec{v}|^2 =\vec{v}\cdot\vec{v}  \implies v\dot{v}
  =\vec{v}\dot{\vec{v}}$. That is $\dot{v}= \left( \frac{\vec{v}}{v} \right)
  \cdot \vec{a}$, hence the rate of change of speed really is the component of
  acceleration in the direction tangent to the particle's trajectory.
  \label{rem:speedAndVel}
\end{remark}
\begin{definition}
  The linear momentum of $P$ is $\vec{p}=m\vec{v}$, its velocity weighted by its
  mass.
\end{definition}
\begin{definition}
  The \emph{angular momentum} of $P$ about a point $\vec{x_0}$ is 
  \[\vec{L}=(\vec{x}-\vec{x_0}) \times \vec{p}\]
  Where the cross product by position generally means \emph{moment}, and in this
  case this is \emph{the moment of linear momentum}. The moment of some
  physical quantity (in our case linear momentum) accounts for how that quantity
  is located in space.
\end{definition}
\begin{definition}[Inertial frame of reference]
  Reference frame with no acceleration.
\end{definition}

\subsubsection{Newton's law}
Next, we look into a review of actual mechanics.
\begin{definition}[Newton's 1st law]
  A point mass (aka particle) with a constant velocity wrt an intertial frame of
  reference, unless it is acted upon by a force.
  \label{def:newton1}
\end{definition}
\begin{definition}[Newton's 2nd law]
  The rate of change of linear momentum is proportional to the force applied,
  \[\vec{F} = \frac{d\vec{p}}{dt}\]
  For a point mass with time-constant mass, i.e. $m(t)=m_0\in\RR$,
  \[\vec{F}=m\vec{a}\]
\end{definition}
\begin{definition}[Newton's 3rd law]
  If an object $A$ applies a force $\vec{F}$ to an object $B$, then $B$ applies
  a force $-\vec{F}$ to $A$. In other words, action and reaction are equal in
  magnitude and opposite in direction.
\end{definition}
It's worth posing on the consequences of Newton's laws,
\begin{enumerate}
  \item $\vec{F}=\vec{0} \implies \vec{p}=p_0\in\RR^n$, i.e. $\vec{p}$ is
    constant. Moreover, if $m$ is constant, then $\vec{v}=v_0\in\RR$, i.e.
    velocity would be constant.

  \item Consider the angular momentum $\vec{L}=\vec{x}\times\vec{p}$, about an
    origin $\vec{x_0}=\vec{0}$. We have 
    \[\frac{d\vec{L}}{dt}=\dot{\vec{x}}\times\vec{p}
    +\vec{x}\times\dot{\vec{p}} = m(\vec{v}\times \vec{v}) +
  \vec{x}\times\vec{F} = \vec{x}\times\vec{F}\]
  As can be seen in the LHS, this is exactly the definition of a moment, in
  particular this is a \emph{moment of force}, or \emph{torque}. Hence the rate
  of change of angular momentum is exactly the torque.

\item Recall that that kinetic energy is defined as $K=\frac{1}{2} mv^2$
  (recall: the integral of force times velocity wrt $t$). Using Newton's second
  law for a constant mass partcile, $m\vec{a}=\vec{F}$, we have
  \[m\dot{\vec{v}}\cdot \vec{v}= \vec{F}\cdot\vec{v}\]
  \[\implies m\frac{d}{dt}\left( \frac{1}{2} \vec{v}\cdot\vec{v} \right) =
  \vec{F}\cdot\vec{v}\]
  By Remark \ref{rem:speedAndVel} we have,
  \[\implies \frac{d}{dt}\left( \frac{1}{2}mv^2 \right) = \vec{F}\cdot\vec{v} \]
  Integrating from $t_0$ to $t_1$, we have
  \[\int_{t_0}^{t_1} \frac{d}{dt}\left( \frac{1}{2} mv^2 \right) dt =
  \int_{t_0}^{t_1} \vec{F}\cdot\vec{v} dt\]
  \[\implies \left[ \frac{1}{2} mv^2 \right]^{t_1}_{t_0} = \Delta K
  \int_{t_0}^{t_1}\vec{F}\cdot\vec{v} dt = \int_C \vec{F}\cdot d\vec{x}\]
  This is the definition of kinetic energy, the force in the direction of
  velocity (and weighed by speed) is the rate of change of this quantity we call
  kinetic energy. The RHS is the work done by $\vec{F}$, which in general
  depends on $C$, but in some cases (conservative fields) it does not.
\end{enumerate}
