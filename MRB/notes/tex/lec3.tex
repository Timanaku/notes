\section{Week 1 - 14 Jan 2022 - Simple pendulum and central forces}
\begin{example}[Simple pendulum]
  A pendulum is a point-mass $m$ attatched to an origin $O$ with an inextensible
  string of length $l$ in a gravitational field $\vec{g}$. The forces acting on
  $m$ are gravity $\vec{g}$, pointing to $-\vec{e_y}$  and the tension of the
  string $\vec{T}$, pointing to $-\vec{e_r}$ pointing towards $O$. We have
  $\vec{e_y}\cdot \vec{e_r}=\cos\theta$.
\end{example}
\begin{proof}[Solution]
  By Newton's second law, $m\ddot{\vec{x}}= m\vec{g} + \vec{T}$, where
  $\vec{x}=l\vec{e_r}$, $\vec{T}=-T\vec{e_r}$, and $\vec{g}=-g\vec{e_y}$.
  Therefore we have 
  \[\dot{\vec{x}} = l\dot{\theta}\vec{e_{\theta}}, \]
  \[\ddot{\vec{x}} = -l\dot{\theta}\vec{e_r} + l\ddot{\theta}\vec{e_{\theta}}.\]
  Hence, we have Newton's second law
  \[m(-l\dot{\theta}\vec{e_r} + l\ddot{\theta}\vec{e_{\theta}}) = -mg\vec{e_y}
  -T\vec{e_r}\]
  We can look at the equation wrt components. We start by $\vec{e_r}$ and
  observe that 
  \[m(-l\dot{\theta}) =-mg\cos\theta - T,\]
  The scalar Newton equation in the $r$ component. Similarly,
  \[m(l\ddot{\theta}) = -mg\cos(\theta+\frac{\pi}{2}) = -mg\sin\theta,\]
  The scalar Newton equation in the $\theta$ component. We can re-arrange the
  $\theta$ component as 
  \[\ddot{\theta} + \omega^2\sin\theta = 0\]
  Where $\omega=\sqrt{\frac{g}{l}}$ is called \emph{frequency}. This is a second
  order ODE, needing an initial condition: $\theta(0)=\theta_0,
  \dot{\theta}(0)=\dot{\theta}_0=\dot{\vec{x}}/l=\frac{v_0}{l}$, as stated in
  the beginning of the solution. Multiplying our ODE by $\dot{\theta}$ and
  integrating we observe that 
  \[\frac{1}{2}\dot{\theta}^2-\omega^2\cos\theta = C\in\RR\]
  This is an equation of conservation of energy.
\end{proof}

\subsection{Chapter 2 - Central force fields}
In this section we consider the motion of a point mass under the influence of a
central force field.
\begin{definition}
  A central force is a force $\vec{F}$ that is directed towards (or away) from a fixed
  location $\vec{x_0}$, called the center of the force. Such a force has a field
  expression 
  \[\vec{F}= f(|\vec{x_0}-\vec{x}|)
  \frac{\vec{x_0}-\vec{x}}{|\vec{x_0}-\vec{x}|}=f(r)\vec{e_r}\]
  Where $\vec{x}$ is the position vector of $m$ and $f$ is a scalar function.
\end{definition}
As per usual, we assume $\vec{x_0}=\vec{0}$.
Prototipical examples include gravitational interactions.

\begin{theorem}
  Angular momentum is conserved in a central force field.
\end{theorem}
\begin{proof}
  By NII we have 
  \[m\ddot{\vec{x}}= F(x)\frac{\vec{x}}{x}\]
  Taking the cross product by $\vec{x}$ we have 
  \[\vec{x}\times (m\ddot{\vec{x}}) = 0\]
  Recall angular momentum $\vec{L}=\vec{x}\times m\dot{\vec{x}}$.
  Differentiating we have 
  \[\dot{\vec{L}}=\dot{\vec{x}}\times (m\dot{\vec{x}}) + \vec{x}\times
  (m\ddot{\vec{x}})=0\]
  Hence it follows that $\vec{L}=\vec{L_0}$. Therefore, motion in a central
  force field conserves angular momentum. In the case where $\vec{L_0}=\vec{0}$,
  we have that $\vec{x}\times\dot{\vec{x}}=0$, i.e. straight line motion. Hence,
  the motion is of the form $\vec{x}=X\vec{n}$ for some constant direction
  vector $\vec{n}$, and so $\ddot{\vec{x}}=\ddot{X}\vec{n}$, and from NII we
  have 
  \[m\ddot{X}\vec{n} = F(X)\vec{n} \implies m\ddot{X}=F(X)\sgn(X)\]
\end{proof}

\begin{theorem}
  Motion in a central force field is in a plane perpendicular to $\vec{L_0}$,
  through the origin.
\end{theorem}
\begin{proof}
  We have $\vec{L_0}=m\vec{x}\times\dot{\vec{x}}$, so
  $\vec{L_0}\cdot\vec{x}=\vec{x}\cdot m\vec{x}\times\dot{\vec{x}}=0$
\end{proof}
Due to this fact, we can safely use plane polar coordinates, defined on the
aforementioned plane. Hence,
\[\vec{L_0} = m\vec{x}\times \dot{\vec{x}} = m(r\vec{e_r}) \times
(\dot{r}\vec{e_r} + r\dot{\theta}\vec{e_{\theta}}) = mr^2\dot{\theta}\vec{e_z}\]
This quantity is constant, hence $r^2\dot{\theta}=h\in\RR$ is constant, and by extension
the acceleration will be simplified to $\ddot{\vec{x}} =
(\ddot{r}-r\dot{\theta}^2)\vec{e_r}$. Hence Newton's equation becomes
$m(\ddot{r}-r\dot{\theta})=F(r)$ in the direction of $\vec{e_r}$, the only
component needed, and in summary we have the system of equations describing the
motion of a mass point under a central force by Newton's second law as,
\[m(\ddot{r}-r\dot{\theta}^2)=F(r)\]
\[r^2\dot{\theta}=h\in\RR\]
And by simplification, it follows that 
\[\ddot{r}- \frac{h^2}{r^3}=\frac{F(r)}{m}\]
\[r^2\dot{\theta}=h\in\RR\]












