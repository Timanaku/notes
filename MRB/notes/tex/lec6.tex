\section{Week 2 - 21 Jan 2022 - Linear theory of stability of orbits}
In the previous lecture we saw how stability of orbits varies according to a
perturbative ODE. Next, we will look into applications of our theory of
stability
\begin{example}
  Consider the central force $F(r)=\frac{GmM}{r^2}$. Do there always exist
  circular orbits?
\end{example}
\begin{proof}[Solution]
  Observe that our condition for circular orbits requires
  $F(r)=\frac{-mh^2}{a^3}$, so we would need 
  \[\frac{-mh^2}{a^3} = \frac{Gm M}{r^2}\]
  \[\iff a= \frac{h^2}{GM}\]
  Henceforth, a circular orbit always exists, for any $h$. To find out if these
  orbits are stable we calculate the term that defined this in our previous
  example,
  \[3F(a)+aF'(a) = -\frac{3GMm}{a^3} + \frac{2GMm}{a^2} = -\frac{GMm}{a^2}<0\]
  Hence these orbits are stable.
\end{proof}

\begin{example}
  What happens when $F(r)=-\gamma r^{-\alpha}$, for $\alpha$ a constant. What is
  the stability of these orbits?
\end{example}
\begin{proof}[Solution]
  We have that $F(r)=\frac{mh^2}{a^3}=\gamma a^{-\alpha}$, hence we must have 
  \[a=\left( \frac{mh^2}{\gamma} \right)^{\frac{1}{3-\alpha}} ; \alpha\neq 3\]
  Hence ciruclar orbits exist. We check stability by simply using the stability
  equation, and see outcomes,
  \[3F(a)+ aF'(a)=-3\gamma a^{-\alpha} + \gamma\alpha a^{-\alpha}\]
  \[= \gamma a^{-\alpha} (-3+\alpha)\]
  Hence we see that for $\alpha>3$, we have $3F(a)+ aF'(a)>0$ giving an unstable
  orbit. Whereas for $\alpha<3$, we have $3F(a)+ aF'(a)<0$ giving a stable
  orbit.
\end{proof}

\begin{example}
  Suppose that a point mass moves under a central force
  $F(r)=-mb(\frac{1}{r^2}+\frac{a}{2r^3})$. Assume that the particle starts at
  $r=a$ with initial speed $v=\frac{b}{a}$ in the direction perpendicular to the
  force (i.e. $\vec{e_{\theta}}$ ). Calculate the equation of the path and
  determine the minimum and maximum distance from the origin.
\end{example}
\begin{proof}[Solution]
  We first find
  $h=r^2\dot{\theta}=a(a\dot{\theta})=a\sqrt{\frac{b}{a}}=\sqrt{ab}$, since
  $a\dot{\theta}$ is exactly the $\theta$-component of velocity, which is
  precisely $v$ as stated in the problem. Next, we apply the equation of motion,
  \[u_{\theta\theta} +u =\frac{-F(\frac{1}{u})}{mh^2u^2} =
  \frac{mb}{mh^2u^2}\left( u^2 + \frac{a}{2}u^3 \right)\]
  \[\implies u_{\theta\theta}+u = \frac{1}{a}+\frac{u}{2}\]
  \[\implies u_{\theta\theta}+\frac{1}{2}u = 1/a\]
  Since we have our equation of motion, we proceed to find ICs. Note that we
  have $\dot{r}=0$ by the initial condition that the initial velocity is in the
  direction $\vec{e_{\theta}}$. Hence $-hu_{\theta}=0$ and it follows that
  $u_{\theta\theta}=0$, so $u(0)=\frac{2}{a}$. The solution to the IVP is then,
  \[u=\frac{2}{a} + A\cos\frac{\theta}{\sqrt{2}} +
  B\sin\frac{\theta}{\sqrt{2}}\]
  With the initial condition we find the solution $u
  =\frac{2}{a}-\frac{1}{a}\cos\frac{\theta}{\sqrt{2}}$.
  It's seen then that $u_{\max}=\frac{3}{a}$ and $u_{\min}=\frac{1}{a}$, which
  corresponds to $r_{\max}=a$ and $r_{\min}=a/3$.
\end{proof}

\subsection{Energy}
Recall that $\frac{d}{dt}\left( \frac{1}{2}mv^2 \right) = \vec{F}\cdot
\dot{\vec{x}}$ is defined as the power delivered by a force. Recall that we
define work done by $\vec{F}$ in the particle moving along the path from $A$ to
$B$, $C_{A\to B}$, as
\[W = \int_{C_{A\to B}}\vec{F} d\vec{x} = \int_{t_A}^{t_B} \left[
\vec{F}(\vec{x}(t)) \cdot \dot{\vec{x}} \right] dt\]
From the integral above, it can easily be seen that in general the work done
depends on the path taken, but there's a particular class of force fields in
which this is not true -- conservative fields. A clear example is that of a
uniform, constant force field. We generalise this in the following theorem.
\begin{theorem}[Conservative fields]
  Suppose that $\vec{F}$ is a vector field over $\RR^2$ or $\RR^3$. The
  following are equivalent.
  \begin{enumerate}
    \item $\oint_C \vec{F}\cdot d\vec{x} =0$ for all closed curves $C$.
    \item $\int_{C_{A\to B}} \vec{F}\cdot d\vec{x}$ is independent of the path
      taken, only depends on endpoints $A,B$.
    \item There exists a scalar field $\phi$ s.t. $\vec{F}=-\nabla \phi$.
    \item $\nabla \times \vec{F}=\vec{0}$.
  \end{enumerate}
  \label{thm:conservativeFieldsEquivalence}
\end{theorem}
\begin{proof}
  \emph{$(1)\implies (2)$} Consider curves $C_1\neq C_2$  joining the endpoint
  $A,B$. Now construct a path $C$ that traverses $C_1$ and then $-C_2$, so that
  the curve $C$ is closed. Then we have 
  \[\oint_C \vec{F}\cdot d\vec{x} =0,\]
  By assumption. Hence
  \[\implies \int_{C_1}\vec{F}\cdot d\vec{x} + \int_{-C_2}\vec{F}\cdot d\vec{x}
  =0\]
  \[\implies \int_{C_1}\vec{F}\cdot d\vec{x} - \int_{C_2}\vec{F}\cdot d\vec{x}
  =0 \]
  And the result then follows.
  
  \emph{$(2)\implies (3)$} By construction, define
  $\phi(\vec{x})=-\int_{C_{0\to x}} \vec{F}(\vec{y}) d\vec{y}$, where the curve
  $C_{0\to x}$ is arbitrary joining $0$ and $\vec{x}$. Consider the directional
  derivative $\vec{u}\cdot\nabla \phi$,
  \[\vec{u}\cdot\nabla \phi= \lim_{h\to 0}\left[ \frac{\phi(\vec{x}+h\vec{u}) -
  \phi(\vec{x})}{h} \right]\]
  \[= \lim_{h\to 0} \left[ \frac{-1}{h}\int_{C_{x\to x+hu}}\vec{F}(\vec{y})
  d\vec{y} \right]\]
  Using a straight line as the curve, we find
  \[\vec{u}\cdot\nabla \phi = \lim_{h\to
  0}\int_0^1\vec{F}(\vec{x}+hs\vec{u})\cdot h\vec{u}
  ds=-\vec{F}(\vec{x})\cdot\vec{u}\]
  Since $\vec{u}$ was arbitrary, it follows that $\vec{F}=-\nabla\phi$.

  \emph{$(3)\implies (4)$} This follows easily since $\vec{F}=\nabla \phi$, so
  that $\nabla \times \vec{F} = \nabla \times \nabla \phi =0$ (basic nabla
  identities). Recall that the meaning of curl is the infinitesimal integral in
  a closed loop, and this being non-zero means that there's a change in the
  gradient. However, in a scalar field this is not possible.

  \emph{$(4)\implies (1)$} Consider a closed curve $C$, enclosing a surface $S$.
  By Stoke's theorem,
  \[\oint_C \vec{F}\cdot d\vec{x} = \int\int_S (\nabla \times \vec{F})\cdot
  \vec{n} dS\]
  Where $\vec{n}$ denotes the normal to the surface $S$. By assumption we have
  $\nabla \times \vec{F}=0$, hence it follows that $\oint_C\vec{F}\cdot
  d\vec{x}=0$.
\end{proof}
\begin{remark}
  The above theorem applies only to simply connected domains. E.g.
  $\RR^2\setminus \{0\}$ would not work. Moreover, it also provides a simple way
  of figuring out whether $\vec{F}$ is conservative: just check $\nabla\times
  \vec{F}=0$.
\end{remark}

