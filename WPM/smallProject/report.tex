\documentclass[11pt]{scrartcl}
\usepackage{evan}
\usepackage[
  backend=biber,
  style=phys,
]{biblatex}
\addbibresource{RSAref.bib}

\title{W\&P: RSA Encryption project}
\author{2480326P}
\date{Fall 2021}

\begin{document}
\section{Introduction}\label{intro}
Exchanging messages secretly and securely has been a challenge for the whole history of
civilization \cite{historyCrypto}. The main challenge has been to securely agree on a
method of \emph{encrypting} messages, i.e. using an agreed scheme of encoding and decoding
messages in a way that any non-wanted third-party can't read these. 

Numerous methods have appeared throughout history \cite{historyCrypto} which required the
sender and the receiver to previously agree on keys. Now called \emph{symmetric key}
cryptography, this method's historical implementations have been proven weak over and over
again \cite{historyCrypto}.  Notwithstanding the drawbacks, given a clever implementation,
symmetric key strategies can be useful and have been even used in practical applications
\cite{AESpaper}. 

However, in 1976 a significant milestone in this problem was achieved when cryptologists
Whitfield Diffie and Martin Hellman published the first system that did not require on
sender and receiver to \emph{previously} agree on keys. Instead they could \emph{exchange}
keys over a public channel securely and create new keys on which they could agree
\cite{diffieHellman}, solving the problem of distributing the keys in a symmetric key
encryption system.

Shortly after and leveraging this work, in 1977, mathematicians and cryptographers Ron
Rivest, Adi Shamir, and Leonard Adleman came up with a method of \emph{asymmetric key}
encryption -- i.e. a method where sender and receiver don't need to agree on using the same keys to
communicate -- that has been used ever since, called RSA named after the authors
themselves \cite{rsaPaper}. In this report the RSA asymmetric key encryption system is presented.

\section{Overview of secure communication}\label{overview}
In a public communication channel two individuals, let us call them Alice and Bob,
exchange messages that anyone can read. Let $S$ be the symbol alphabet that Alice and Bob
agreed on, which may be the english alphabet. Let $m$ be the original message Alice writes
for Bob, so $m=s_1 s_2 s_3\dots s_n$, where $s_i\in S$. This could be, for example, an
english sentence.  Due to scope of this report, we will ignore technical implementation
details like source coding and channel coding (for compression and error correction). 

In order to communicate through the channel, the message $m$ must be transformed to a
numeric representation (so that a signal can be sent) in a bijective way -- otherwise it
would be impossible for Bob to retrieve the message.  Let the sent message be $m^*=a_1 a_2
a_3 \dots a_n$, where $a_i\in\ZZ$ and $m^*=\phi (m):=\phi(s_1)\phi(s_2)\dots\phi(s_n)$.
The bijective mapping $\phi:S\to\ZZ$ is called character encoding \cite{characterEncoding}, and
for the majority of modern-day computer applications the American Standard Code for
Information Interchange (ASCII) is used as the character encoding strategy. After Bob
receives the message and decodes it by using $\phi^{-1}$, the communication process
concludes.

The challenge is introduced if a third-party intercepts $m^*$ (hence the name of public
channel), which by using $\phi^{-1}(m^*)$ they can read the message. Under the RSA
encryption system, this challenge is solved by introducing a public and private keys
system. Let $A_{pub},A_{priv}\in P$ denote Alice's public and private keys, and
$B_{pub},B_{priv}\in P$ denote Bob's public and private keys, respectively. These keys, as
their names suggest, are kept public and private, respectively. Here $P$ represents the
set of possible keys, whose elements are in practice pairs of integers. 

RSA works in the following manner. Let $m$ be the message Alice sends to Bob, as above.
Alice undergoes an extra step before sending the message, a map $m'=\psi(m^*,
B_{pub})=a'_1 a'_2 a'_3 \dots a'_n$ where $a'_i\in\ZZ$ and $\psi: \ZZ\times P \to \ZZ$ is
a non-bijective mapping. The message $m'$ is sent through the public channel, and Bob
\emph{decrypts} the message by using the same mapping but different parameters
$m^*=\psi(m', B_{priv})=a_1 a_2 a_3 \dots a_n$. Then $m=\psi^{-1}(m^*)$, and the
communication process concludes. Note that $B_{priv}$ was kept secret throughout, and that
the \emph{only} (more on this later) way to decrypt the message is with $B_{priv}$, hence
the message is secure to third-party interception. Bob can send a message to Alice by
following the same process, using $A_{pub}$.

\section{Background and Theorems}\label{background}
The above was an explanation of the communication process and an overview of what
encryption by asymmetric key is. In this section, before we explain the RSA algorithm, we
will introduce theorems that will be necessary to back claims done in its description.

\begin{definition}[Euler's Totient Function]
  Let $n$ be a natural number. Euler's Totient function, $\phi (n)$, is defined as the
  number of positive integers less than or equal to $n$ that are coprime to $n$.
  \label{totient}
\end{definition}

\begin{theorem} [Fermat's Little Theorem]
  Let $p$ be a prime integer and $a\in\ZZ$ s.t. $\gcd (a,p)=1$. We have
  \[a^{p-1}\cong 1 \mod p\]
  \label{fermat}
\end{theorem}

\begin{theorem}[Euler's Theorem]
  If $n$ and $a$ are integers with $\gcd (a,n)=1$, we have
  \[a^{\phi (n)} \cong 1 \mod n\]
  \label{euler}
\end{theorem}

\begin{lemma}
  Let $p$ be a prime number. Then $\phi(n) = n-1$.
  \label{trivialP}
\end{lemma}
\begin{lemma}
  If $p$ and $q$ are prime, then $\phi (pq)=\phi(p) \phi(q)$.
  \label{compositePhi}
\end{lemma}
Proofs to the above theorems and lemmas can be found in \cite{liebeck},\cite{maxey}.

\section{The Algorithm}\label{algo}
In Section \ref{overview} we explained the private key approach used by RSA. In this
section, we will specify this process. In particular, we will define how Bob generates
their $B_{pub}, B_{priv}$ keys, and how the non-bijective mapping $\psi$ is defined. Hence
we will specify how Alice is able to encrypt a message directed to Bob, and how Bob is
able to decrypt it.

In order to compute Bob's keys (a process called key generation), they must first choose
two prime numbers. Let $p, q$ denote two large prime integers, of Bob's choice. Then, by using
Euler's Totient function and Lemma \ref{compositePhi} we define,
\[n:=pq\]
\[m:=\phi(pq) = (p-1)(q-1)\]
Then we aim to find $e$ s.t. $1<e<m$ with $\gcd (e, m)=1$. This can be done by choosing
some odd number, and iteratively adding $2$ and checking $\gcd (e, m)=1$ at every iteration.
The addition by $2$ is because we know even numbers won't be coprime to $m$ (a product of
two even numbers), hence it doesn't make sense to check for them. We must have $e$ and $m$
coprime to use Theorem \ref{euler}.  Then we define $d:=e^{-1}\mod m$, the multiplicative
inverse of $e$.  Then, we define Bob's keys as $B_{pub}=\left\{ n, e \right\}$ and
$B_{priv}=\left\{ n, d \right\}$. At this point, the reader might see how, by finding $n$
factors, an attacker could unravel all the numbers we just defined.

Once Alice has access to Bob's public key $B_{pub}$, she is able to send a message by
using $\psi$ introduced in Section \ref{overview}, which is defined as follows. Let $a\in
S$, $a' = \phi(a)$ -- encoding the character with the agreed scheme--, and $P_{r}\in P$ some
key pair $P=\{n',k'\}$, may it be public or private. Then we have
\[\psi (a', P_{r}) := (a')^{P_{r}^{(2)}}\mod P_{r}^{(1)} = (a')^{k'}\mod n'\]
Note that we're using the superscript notation to refer to the element in the $i$-th
position of the key pair. Hence, for Alice to encrypt a message for Bob, applying the
mapping character by character (or by symbol blocks \cite{paddingCrypto}, which is
outside the scope of this report) obtains the mapping 
\[a_i' = \psi(a_i, B_{pub}) = a_i^{B_{pub}^{(2)}}\mod B_{pub}^{(1)} = a_i^e \mod n\]
By following the notation introduced in Section \ref{overview}. . Once Bob receives the
message, as explained before, they just use the same mapping but with the private key,
\[a_i = \psi(a_i', B_{priv}) = a_i'^{B_{priv}^{(2)}}\mod B_{priv}^{(1)} = a_i'^d \mod n\]
Finally, using the inverse encoding scheme, $s_i=\phi (a_i)$, and the communication
process concludes. Note that this expression follows from Theorem \ref{euler}, since $a_i=
a_i'^d\mod n = a_i^{ed}\mod n = a_i^{lm + 1}\mod n$ for some integer $l$. This equality
follows from the fact that $e$ is the multiplicative inverse of $d\mod m$. Hence $a_i=
a_i (1)^l\mod n$. This process generalises for 2-way communication between Alice and Bob
after Alice generates their private and public keys in the same fashion but with $p,q$ of
their own choice.

\section{Limitations and vulnerabilities}\label{limitations}
Even though the mapping $\psi$ is non-bijective, the nature of the RSA algorithm,
language, and the integers leave open doors to the inverse problem of finding prime
factors of large numbers.


Due to predictability of language, one can potentially break RSA codes with frequency
analysis \cite{cypherTextAttack}. Under this strategy, one uses the predictability of
language to look for patterns in the encrypted data. Due to the fact that RSA is a
deterministic algorithm (i.e. two equal numbers are encrypted to the same number), one can
potentially exploit this feature to decrypt a message. However, defense against these kind
of attacks are used in commercial applications. For instance, the primal example of this
is the padding technique \cite{paddingCrypto}, where gibberish data is added at the
beginning, middle, and end of a message before encrypting, to render frequency analysis
useless.


Nevertheless, there exists a more dangerous approach for breaking RSA. This is converting it
into the factoring problem. As seen in previous sections, the public key contains the
semiprime integer $n$. This is what enables the asymmetric cryptography communications
between Alice and Bob, but it also leaves room for exploits. The key generation algorithm
of the RSA turns the public number $e$ into the private number $d$, which allows us to
decipher messages. Hence, if we're able to factor $n$, we can get the private key. This
problem, however, is not an easy one to solve for very large $n$ \cite{factorization}
\cite{tale2009}. 

Although it's not been proven to be infeasable, by using a brute-force approach,
trying to factor large numbers -- which in commercial applications are in the order of
$2^{1024}$, a number with more than 100 digits -- with current computing capabilities
would take around $10^6$ years to solve \cite{factorization} \cite{tale2009}. However,
there are threats to the infeasability of this problem, naming quantum computing. The Shor
algorithm, named after Peter Shor, demonstrates that, given an ideal quantum computer, it
is efficient to factor large integers \cite{shorAlgorithm}. Even though RSA faces a threat with quantum
computing, as Shor put it himself \cite{shor2020}, the security of the internet is not \emph{too}
threatened by it, since other encryption algorithms exists.

\section{Conclusions}
In conclusion, RSA cleverly leveraged centuries of knowledge of number theory to provide a
solution to secure communications over public channels. In this report we presented how
this is possible, how the \emph{theoretical} process works, and what limitations it has.
Notwithstanding the current threats and potential obsolence of the algorithm, the RSA
algorithm has provided a way to grow a secure internet, allow online payments possible,
and give the possibility to people like the reader to securely send cat memes over email.

\newpage

\printbibliography
\end{document}
