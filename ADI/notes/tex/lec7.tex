\section{Lecture 7 - 26 Oct 2021}
\subsection{Local Extrema, Rolle's Theorem}
\begin{definition}
  Fix $f:S\to\RR$. Let $c\in S$. Then 
  \begin{enumerate}
    \item $f$ has a local maxima in $c$ if there exists an interval $I\subset S$ around
      $c$ s.t. if $x\in I\cap S$ then $f(x)\leq f(c)$.
    \item $f$ has a local minima in $c$ if there exists an interval $I\subset S$ around
      $c$ s.t. if $x\in I\cap S$ then $f(x)\geq f(c)$.
    \item $f$ has a local extremum at $c$ if $f$ has either a local minima or local maxima
      at $c$.
  \end{enumerate}
  \label{<+label+>}
\end{definition}


\begin{lemma}
  Let $f:(a,b)\to\RR$ be differentiable at $c\in(a,b)$ and let $c$ be a local extremum for
  $f$. Then $f'(c)=0$.
  \label{<+label+>}
\end{lemma}
\begin{proof}
  Let us proceed by contradiction and assume $c$ is a local maxima (without loss of
  generality). Let $I=(s,t)$ containing $c$ s.t. $a<s<c<t<b$. If $x\in I$, then $f(x)\leq
  f(c)$ and $f'(c)\neq 0$. Either $f'(c)>0$ or $f'(c)<0$. By definition, if $f'(c)>0$, for
  $\eps=\frac{f'(c)}{2}>0$ there exists $\delta>0$ s.t. for $(c-\delta, c+\delta)\subseteq
  I$ and $|x-c|<\delta$ we have
  $\left|\frac{f(x)-f(c)}{x-c}-f'(c)\right|<\frac{f'(c)}{2}$. Notice that for $x\in
  (c-\delta, c+\delta)$ we have the previous inequality,
  \[-\frac{f(x)-f(c)}{x-c}+ f'(c) < \frac{f'(c)}{2}\]
  \[ \iff \frac{f'(c)}{2} < \frac{f(x)-f(c)}{x-c}\]
  Recall that $f(c)\geq f(x)$ since $c$ is a local maxima by assumption. Choose $x>c$ so
  that $x-c>0$, and note that $\frac{f(x)-f(c)}{x-c} \leq 0$ but we had $f'(c)>0$ by
  assumption, a contradiction. For $f'(c)$, the argument is analogous, with
  $\eps=\frac{-f'(c)}{2}$.
\end{proof}
However, note that this statement is one-sided
\begin{example}
  Consider $f(x)=x^3$ and note that $f'(0)=0$. However, for any $x<0$, $f(x)<f(0)$ and for
  any $x>0$, $f(x)>f(0)$ since $f$ is strictly increasing. 
\end{example}

\begin{theorem}[Rolle's Theorem]
  Let $a<b$ and let $f:[a,b]\to\RR$ be a continuous s.t. $f$ is differentiable in $(a,b)$
  and $f(a)=f(b)$. Then $\exists c\in (a,b)$ s.t. $f'(c)=0$.
  \label{<+label+>}
\end{theorem}
\begin{proof}
  By the Extreme Value Theorem (EVT) there are $u,v\in[a,b]$ s.t. 
  \[f(u)\leq f'(x) \leq f(v) \quad \forall x\in [a,b]\]
  Then by the above lemma, $f'(u)=0$. If $u\in (a,b)$ then $c=u$.
  Similarly, we have $f'(v)=0$. If $v\in (a,b)$ then $c=v$. Otherwise, $u,v\in \left\{
  a,b \right\}$. For this case, notice that EVT still applies so $f(u)\leq f(x)\leq f(v)$
  and $f(a)=f(b)$, then $f(u)=f(x)=f(v)$ for all $x\in [a,b]$, i.e. $f$ is a constant
  function and hence $f'(x)=0$ for all $x\in[a,b]$. 
\end{proof}
Note that the codomain of our function above is real, even though the statement itself
doesn't use orderings at all. There're counterexamples to Rolle's theorem in
complex-valued functions!
\begin{example}
  Take $\gamma:[0,2\pi]\to\CC:t\to \cos(t)+i\sin(t)$. By definition,
  $|\gamma'(t)|=|i\gamma(t)| =1$ for all $t$ in the domain and $\gamma(0)=\gamma(2\pi)$.
  Then it will never satisfy Rolle's Theorem.
\end{example}

\subsection{Mean Value Theorem}
A generalisation of Rolle's Theorem.
\begin{theorem}[Mean Value Theorem]
  Let $a<b$. Let $f:[a,b]\to\RR$ be continuous and differentiable in $(a,b)$. Then there
  exists $c\in(a,b)$ s.t. 
  \[f'(c)= \frac{f(b)-f(a)}{b-a}\]
  \[\left( f(b)=f(a)+f'(c)(b-a) \right)\]
  \label{<+label+>}
\end{theorem}
\begin{proof}
  Let $g:[a,b]\to\RR:t\mapsto f(t)-kt$ for some $k\in\RR$, s.t. $g(a)=g(b)$.
  \[f(a)-ka = f(b)-kb \iff k= \frac{f(b)-f(a)}{b-a}\]
  By Rolle's Theorem, there is some $c\in(a,b)$ s.t. $g'(c)=0$. Note
  \[g'(c)=f'(c)-k=0 \iff f'(c)=k=\frac{f(b)-f(a)}{b-a}\]
\end{proof}

\begin{example}
  Let $f:[a,b]\to\RR$ be continuous and $f'(t)=0$ for all $t\in(a,b)$. Then we claim $f$
  must be a constant function. 
\end{example}
\begin{proof}[Solution]
   Assume $f$ is not constant. Then $\exists s<t$ s.t. $f(s)\neq f(t)$. By the Mean Value
   Theorem (MVT), $\exists c\in(s,t)$ s.t. 
   \[f'(c)=\frac{f(s)-f(t)}{s-t}\]
   Then it follows that $f'(c)\neq 0$, a contradiction.
\end{proof}


\begin{exercise}
  Let $f:[a,b]\to\RR$ be differentiable and $|f'(x)|\leq M$ for all $x\in[a,b]$. Show that
  for all $x,y\in[a,b]$ we have 
  \[|f(x)-f(y)|\leq M|x-y|\]
  So $f$ is $M$-Lipschitz.
\end{exercise}
\begin{proof}[Solution]
  Suppose there exist $x<y\in[a,b]$ s.t. $|f(x)-f(y)|>M|x-y|$. Then, by MVT there exists
  $c\in (x,y)$ s.t. $f'(c)=\frac{f(y)-f(x)}{y-x}$, which implies $|f'(c)|>M$, a
  contradiction.
\end{proof}
\subsection{Monotonicity}
\begin{definition}
  Say $f:S\to\RR$ is monotonically increasing if for all $x\leq y$ in $S$, we have
  $f(x)\leq f(y)$.
  Say $f:S\to\RR$ is monotonically decreasing if for all $x\leq y$ in $S$, we have
  $f(x)\geq f(y)$.
  \label{<+label+>}
\end{definition}

\begin{exercise}
  Use the MVT to prove the Monotonicity Theorem.

  Take $a<b$ and let $f:[a,b]\to\RR$ be continuous and differentiable in $(a,b)$. The
  following hold
  \begin{enumerate}
    \item If $f'(x)=0$ for all $x\in(a,b)$, then $f$ is constant on $[a,b]$.
    \item If $f'(x)\geq 0$ for all $x\in(a,b)$, then $f$ is monotonically increasing.
    \item If $f'(x)\leq 0$ for all $x\in(a,b)$, then $f$ is monotonically decreasing.
    \item If $f'(x)>0$ for all $x\in(a,b)$, then $f$ is strictly increasing.
    \item If $f'(x)<0$ for all $x\in(a,b)$, then $f$ is strictly decreasing.
  \end{enumerate}
\end{exercise}
\begin{proof}[Solution]
  Note that (1) is already done in the previous section. Note that (3) follows from (2) and (5) follows from
  (4). Hence we prove only (2) and (4). 

  To prove (2), let $f'(x)\geq 0$ for any $x\in (a,b)$ and assume there exists
  $x<y\in[a,b]$ s.t. $f(x)>f(y)$. Then by MVT we have $f'(c)=\frac{f(y)-f(x)}{y-x}$ for
  some $c\in(x,y)$. Note that $f'(c)<0$. This is a contradiction 

  To prove (4), let $f'(x)>0$ for any $x\in(a,b)$ and assume there exists
  $x<y\in[a,b]$ s.t. $f(x)\geq f(y)$. Then there exists $c\in (x,y)$ s.t.
  $f'(c)=\frac{f(y)-f(x)}{y-x}$. Note that $f'(c)\leq 0$, a contradiction.
\end{proof}

\begin{exercise}
  Show $\sin(x)\leq x$ for all $x\in [0,\infty)$.
\end{exercise}
\begin{proof}[Solution]
  Let us define $f:[0,\infty)\to\RR:x\mapsto x-\sin(x)$. We claim $f$ is monotonically
  increasing. Note $f'(x)=1-\cos(x)$ and since $|\cos(x)|\leq 1$, the result follows. 
\end{proof}

\begin{exercise}
  Use induction on $n$ to show 
  \[\exp x > \sum_{k=0}^{n}\frac{x^k}{k!}\]
  For all $x>0$. Deduce that, for $e:=\exp 1$ for all $n\geq 1$, we have
  \[e>\sum_{k=0}^n \frac{1}{k!} \]
\end{exercise}
\begin{proof}[Solution]
  Note that for $n=0$, it follows that $exp(x)>0$, a true statement. Then, assume that
  $\sum_{k=0}^n \frac{x^k}{k!}<\exp(x)$ for all $x>0$. Let us define the function
  \[f:(0,\infty)\to\RR:x\mapsto \exp(x) \sum_{k=0}^{n+1} \frac{x^k}{k!}\]
  We claim that $f$ is strictly increasing. We have 
  \[f'(x)= \exp x - \sum_{k=0}^{n+1}\frac{x^{k-1}}{(k-1)!} = \exp x -
  \sum_{k=0}^n\frac{x^k}{k!}\]
  Hence by induction, for any $n\geq 0$ the claim above holds, and $f(x)>0$ for all $x>0$.
\end{proof}


\begin{exercise}
  Find all integers $1\leq n < m$ s.t. 
  \[n^m= m^n\]
  In the pdf we end up showing that $2.5<e<3$.
\end{exercise}
\begin{proof}
  Consider the function $f:(1,\infty)\to\RR: t\mapsto \frac{t}{\log t}$. First, observe
  that 
  \[f'(t)=\frac{1}{\log t}-\frac{1}{\log^2 t}\]
  Hence $f'(t)>0 \iff t>\exp(1)=e$ and $f'(t)<0 \iff t<e$. Therefore we show that $f$ is,
  by the monotonicity theorem, strictly decreasing on $(1,e]$ and strictly increasing on
  $[e,\infty)$. This implies that $f(e)$ is a global minima, i.e. $f(x)>f(e)$ for any
  $x\in (1,\infty)\setminus \left\{ e \right\}$.  

  Next, we aim to find positive integers s.t. $m\log n = n\log m$. First, note that $n\neq
  1$, since if that was the case, then $m\log n=0$ while $n\log m\neq 0$. Moreover, we
  have that this equation implies that $f(n)=f(m)$, and by the previous argument this is
  true if and only if $n<e<m$. We know $e<4$, so $n=2$ or $n=3$. Note that for $n=2$,
  $m=4$ is the only solution. Next, assume $3<e$, so there exists a number $x>e$ s.t.
  $f(x)=f(3)$. Notice that $x<4$ since otherwise $f(x)>f(4)=f(2)$, since $f$ is strictly
  increasing after $e$, and $f(2)>f(3)$ since $f$ is strictly decreasing before $e$, hence
  it would follows that $f(x)>f(3)$. Therefore, $3<e<x<4$. Hence $x$ cannot be an integer.
  Therefore, the only solution is $n=2,m=4$.

  Note that $n^m=m^n \iff n\log m = m\log n$, so the solution to the proposed problem
  follows from the previous paragraph. 

\end{proof}<++>
