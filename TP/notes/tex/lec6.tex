\section{Week 2 - 21 Jan 2022 - Sequences}
\begin{proposition}
  In a metric space $(X,d)$ we have $x_n\to a$ as $n\to\infty$ (for a sequence
  $(x_n)$ in $X$ and $a\in X$) iff 
  \[d(x_n,a) \to 0 : n\to\infty.\]
  \label{prop:convergenceMetric}
\end{proposition}
\begin{proof}
  TODO. (Unpack the definition of convergence)
\end{proof}

Next, we look into special cases and examples. 
\begin{proposition}
  Let $(x_n)$ be a sequence in a metric space $(X,d)$, and take $a\in X$. If $X$
  is either discrete -- i.e. $d$ is the discrete metric -- or finite -- i.e. $X$
  is a finite set --, then $x_n\to a$ iff $\exists N\in\NN : x_n=a \forall n>N$.
  \label{<+label+>}
\end{proposition}
\begin{proof}
  If $\exists N : x_n=a \forall n>N$, then $\forall \eps>0$, 
  \[n>N \implies d(x_n,a)=0<\eps.\]
  For the converse, assume $X$ is discrete (not used before). Since $x_n\to a$
  as $n\to\infty$, then $\exists N$ s.t. $d(x_n,a)< 1$ for all $n>N$. But $d$ is
  the discrete metric so $d(x_n,a)=1$ iff $x_n\neq a$. Hence, $n>N$ implies
  $x_n=a$.

  If $X$ is finite, let $X=\{x_1,\cdots, x_n\}$ and let $r=\min_{i\neq j}
  d(x_i,x_j)>0$. The argument is then similar to the discrete case, replacing
  $1$ by $r$.
\end{proof}

We can look into concrete examples of less-intuitive sequences given our
generalisation.
\begin{proposition}
  Let $(x_1,y_1),(x_2,y_2)\cdots$ be a sequence in $\RR^2$ with metric
  $d_{\infty}$. Then  $(x_n,y_n)\to(a,b)\in\RR^2$ iff $x_n\to a$ and $y_n\to b$
  as $n\to\infty$.
  \label{prop:convergenceR2}
\end{proposition}
\begin{proof}
  Suppose $(x_n,y_n)\to(a,b)$. That is 
  \[d_{\infty}\left( (x_n,y_n),(a,b) \right)\to 0.\]
  Note $d_{\infty}\left( (x_n,y_n),(a,b) \right) = \max (|x_n-a|,|y_n-b|)\geq
  |x_n-a|\geq 0$. Hence by the Sandwich theorem it follows that $x_n\to a$,
  and similarly for $y_n\to b$. 

  Conversely, suppose $x_n\to a$ and $y_n\to b$, i.e. $|x_n-a|\to 0$ and
  $|y_n-b|\to 0$. Then $0\geq d_{\infty} \left( (x_n,y_n),(a,b)
  \right)=\max(|x_n-a|,|y_n-b|) \geq |x_n-a|+|y_n-b|\to 0$. Hence again by the
  Sandwich theorem, it follows that $(x_n,y_n)\to(a,b)$.
\end{proof}
\begin{lemma}
  If $d,d'$ are two metrics on $X$, and suppose $\exists c>0$ s.t. $d(x,y)\leq
  Cd'(x,y)$ for all $x,y\in X$. Then for a sequence $(x_n)$ in $X$ and $a\in X$,
  we have that 
  \[x_n\to a \text{ in $(X,d')$}\implies x_n\to a \text{ in $(X,d)$}\]
  \label{<+label+>}
\end{lemma}
\begin{proof}
  Suppose $x_n\to a$ in $(X,d)$. Then 
  \[0\geq d(x_n,a)\leq Cd'(x_n,a)\to 0\]
  And by Sandwich theorem, the result follows. 
\end{proof}

\begin{corollary}
  If $d,d'$ are strongly equivalent metrics on $X$, then 
  \[x_n\to a \text{ in $(X,d)$}\iff x_n\to a \text{ in $(X,d')$}\]
  \label{<+label+>}
\end{corollary}
\begin{proof}
  TODO. Apply the lemma twice.
\end{proof}

\begin{corollary}
  Let $(x_n,y_n)$ be a sequence in $\RR^2$. Then 
  \[(x_n,y_n)\to (a,b) \text{ in $(\RR^2,d_1)$} \iff (x_n,y_n)\to (a,b) \text{ in $(\RR^2,d_2)$}\]
  \[\iff (x_n,y_n)\to (a,b) \text{ in $(\RR^2,d_{\infty})$}\]
  \label{<+label+>}
\end{corollary}
We then look into an example in $\RR^2$ where a sequence converges with $d_2$
but not with the railway metric $D$ (a sequence of points that is not colinear
with the origin and the converging point). Moreover, we see that the given
sequence does not converge at all with $D$.

\begin{example}
  Recall the metrics $d_1, d_{\infty}$ on $C[0,1]$. We saw that $d_1(f,g)\leq
  d_{\infty}(f,g)$ for all $f,g\in C[0,1]$, hence $f_n\to y$ with $d_{\infty}$
  implies $f_n\to y$ in $d_1$. However, note that for $f_n(t)=t^n$ and $y=0$, we
  have that $d_1(f,g)\to 0$ while $d_{\infty}(f_n,y)=1$ for all $n$.
\end{example}
