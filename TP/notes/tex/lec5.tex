\section{Week 2 - 19 Jan 2022 - Function spaces (an example of homeomorphism)
and Sequences}
\begin{proposition}
  We have that $d_1(g,f)\leq d_{\infty}(g,f)$ for any $f,g\in C[0,1]$, but these
  metrics are not strongly equivalent.
  \label{prop:C01metricsNotEq}
\end{proposition}
\begin{proof}
  Let $f,g\in C[0,1]$. By definition $|f(t)-g(t)|\leq d_{\infty}(f,g)$, for any
  $t\in [0,1]$. Moreover, we have that 
  \[d_{1}(f,g)=\int_{0}^1 |f(t)-g(t)|dt \leq \int_0^1d_{\infty}(f,g)dt =
  d_{\infty}(f,g).\]

  In order to show the not equivalence, we claim there exists function
  $y,f_1,f_2,\dots\in C[0,1]$ s.t. $y(t)\neq f_n(t)$ for all $t\in [0,1]$ and
  all $n\in\NN$, with 
  \[\frac{d_{1}(f_n,y)}{f_{\infty}(f_n,y)}\to 0, n\to\infty.\]
  The fact that this implies non-strong-equivalence is an exercise in Exercise
  Sheet 2. We take this as true, in order to proceed. We find the function
  $f_n(t)=t^n$ and $y(t)=0$. Hence, observe that $d_{\infty}(f_n,g)=1$ for all
  $n\in\NN$, whereas we have 
  \[d_1(f,g)= \int_0^1 |f(t)-g(t)|dt = \int_0^1t^n dt = \frac{1^{n+1}}{n+1}\]
  Which clearly tends to $0$. Hence the metrics $d_1, d_{\infty}$ are not
  strongly equivalent.
\end{proof}

What do $d_1,d_{\infty}$ on $C[0,1]$ have to do with the metrics with same
notation on $\RR^n$? We will answer this question with the next proposition.
Recall that $B(X)$ is the set of bounded function $X\to\RR$ for an non-empty set
$X$.
\begin{proposition}
  If $X=\{1,\dots, n\}$, the map $\phi:B(X)\to\RR^n$ given by
  $\phi(f)=(f(1),\dots,f(n))\in\RR^n$ is a bijection. Moreover,
  $d_{\infty}(\phi(f),\phi(g))= d_{\infty}(f,g)$ for any $f,g\in B(X)$.
  \label{<+label+>}
\end{proposition}
\begin{proof}
  Note that $\phi(f)=\phi(g)\implies \phi(f)_i=\phi(g)_i$ for all
  $i\in\{1,\dots,n\}$. Hence it follows that $f(i)=g(i)\forall i$, and this is
  precisely $f=g$. Hence $\phi$ is an injection.
  
  On the other hand, let $x=(x_1,\dots,x_n)\in\RR^n$, and define $f$ by
  $f(i)=x_i$ for $i\in X$. Note taht since $X$ is a finite set, $f(i)$ is well
  defined for all $i$ -- i.e. the minimum and the maximums exist, or in other
  words, $f$ is bounded below and above. It then follows that $f\in B(X)$. By
  construction, $\phi(f)=(f(1),\dots, f(n)) = (x_1,\dots, x_n)=x$. Since $x$ was
  not constained, it follows that $\phi$ is a surjection. Therefore it follows
  that $phi$ is a bijection (in fact, a linear bijection of vector spaces).

  Following, for $f,g\in B(X)$ we have
  \[d_{\infty}(\phi(f),\phi(g))= \sup_{1\leq i\leq n} |\phi(f)_i-\phi(g)_i|=
  \sup_{1\leq i \leq n} |f(i)-g(i)| = d_{\infty}(f,g)\]
\end{proof}
\begin{remark}
  The above is an example of a homeomorphism (think of isomorphisms of vector
  spaces that preserve distances). Later on we will look into detail the general
  notion. In fact we will see that these homeomorphisms in general are just
  isomorphisms of \emph{topological spaces}, and thus must preserve topological
  properties (of which the metric is just a special case). The particular case
  where the distance is preserved is called \emph{isometry}.
\end{remark}

For functions in $C[0,1]$ (a special case of the general $B(X)$, where in this
case $X$ is infinite -- in fact compact closed -- and with the added property
that the functions are continuous), this space has metrics $d_k$ defined as
follows
\[d_k(f,g)= \int_0^1\left( |f(t)^k - g(t)^k| \right)^{1/k} dt,\]
For $k\in\NN$, and they are related to the metrics $d_k$ in $\RR^n$ in a
similar way.

\subsection{Sequences}
One of the goals with metric spaces is to generalise real analysis to analysis
on any metric space. Recall that a sequence $(x_n)_n$ in a non-empty set $X$ is
a list of elements $x_1,x_2,\dots$ of $X$ indexed by $\NN$. Recall the following
defintion in $\RR$,
\begin{definition}
  If $(x_n)$ is a sequence on $\RR$, we say $x_n$ converges to $a\in\RR$ if the
  following is satisfied,
  \[\forall\eps>0 \exists N\in\NN : n>N \implies |x_n-a|<\eps.\]
  In which case we write $\lim_{n\to\infty}x_n=a$.
\end{definition}
Notice that $|x_n-a|=d(x_n,a)$ where $d$ is the usual $\RR$ metric. We see the
next more general definition of a convergent sequence.
\begin{definition}
  Let $(X,d)$ be a metric space, and let $(x_n)$ be a sequence in $X$. We say
  $x_n$ converges to $a\in X$ if $\forall \eps>0\exists N\in\NN$ s.t. 
  \[n>N \implies |x_n-a|<\eps.\]
  In this case we call $a$ the limit of the sequence $(x_n)$.
  \label{def:metricSequence}
\end{definition}
\begin{proposition}
  A sequence in a metric space has at most one limit point.
  \label{prop:uniqueLimitMetricSpace}
\end{proposition}
\begin{proof}
  Let $(X,d)$ be a metric space and $(x_n)$ a sequence in $X$. Assume $a,b\in X$
  are both limit points of $x_n$. We claim $a=b$. It is enough to show that
  $d(a,b)<\eps$ for any $\eps>0$.

  Pick $\eps>0$. Then $\exists N\in\NN : d(x_n,a)<\eps/2$ for any $n\geq N$.
  Similarly, $\exists N'\in\NN : d(x_n,b)<\eps/2$ for any $n\geq N'$. Let $n\geq
  \max (N,N')$, and observe then 
  \[d(a,b)\leq d(a,x_n)+d(x_n,b) < \eps\]
  As required.
\end{proof}<++>
