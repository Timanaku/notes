\section{Week 0 - 7 Jan 2022 - Introduction}
\subsection{Intro}
Topology: Study of properties of spaces (or geometric objects) that do not
change under continuous defomations. These ideas are of importance to many
fields such as analysis, topology, and branches of geometry.

First we will learn about metric spaces: Sets with a distance function. We will
also consider functions between metric spaces. After this, we will pass to the
(more general) concept of topological space (metric spaces are an example of).

\subsection{Lecture 1 - Background}
\subsubsection{Sets and function}
\begin{definition}
  A \emph{set} is a collection of objects, or elements. We say two sets are
  equal if and only if they have precisely the same elements. We say $a\in
  A$ to mean $a$ is an element of $A$. If $A,B$ are sets, we say $A\subseteq B$ to
  mean that $A$ is strictly contained in $B$, and $A\subset B$ to mean that $A$
  is merely contained in $B$ (but may be equal). 
  \label{def:set}
\end{definition}
An example worth remarking is the rational numbers $\QQ$, which can be defined
as the equivalence classes of $\{a/b : a,b\in\ZZ, b\neq 0\}$, with equivalence
$a/b\sim c/d \iff ad=bc$.
\begin{definition}
  Let $A,B$ be sets. The Cartesian product of $A,B$ is the set 
  \[A\times B = \{(a,b) : a\in A, b\in B\}\]
  \label{def:cartesianProduct}
\end{definition}
\begin{definition}
  Let $X,Y$ be sets. A function $f:X\to Y$ is a rule that assigns elements of
  $X$ to elements of $Y$, by $f(x)\in Y$. The \emph{domain} of $f$ is a subset
  of $X$, the \emph{codomain} of $f$ is $Y$, and the \emph{image} of $f$ is
  $\Img(f)=f(\dom f)$. For any subset $B\subset Y$, the \emph{pre-image} (or
  \emph{inverse image}) of $B$ under $f$ is $f^{-1}(B)=\{x\in X:f(x)\in B\}$,
  which does not require invertibility of $f$.
  \label{def:function}
\end{definition}
\subsubsection{Real numbers}
While being the most important set for this course, the real numbers are quite
tricky to define. We may do so by using many approaches, like using the
equivalence classes of all cauchy sequences from $\QQ$, or by using the
infinite/finite decimal expansions and then quotiening by equivalence relation
of the ambiguity that for periodic $9$s from the $n$th decimal, it's equivalent
to $1$ followed by $0$s periodic in the $(n-1)$th decimal. This set satisfies
the axioms of an ordered field. Moreover, it also satisfies the completeness
axiom.
\begin{definition}[Least upper bound axiom]
  Every nonempty subset of $\RR$ which is bounded above has a least upper
  bound (or supremum). That is, consider $A\subset\RR$ such that there exists
  $v\in\RR$ with $a\in A \implies a\leq v$. Then, by this axiom, $\exists
  u\in\RR$ s.t. if $v\in\RR$ is an upper bound for $A$, then $u\leq v$.
  \todo{what happens if we dropped this axiom? Does there any interesting
  happen?}
  \label{axiom:leastUpperBound}
\end{definition}
This axiom follows from the intuition that there exists no gaps in the real
numbers.
\begin{lemma}
  There exists a number $x\in\RR$ s.t. $x^2=2$.
  \label{lem:sqrt2}
\end{lemma}
\begin{proof}
  Consider the set $A=\{x\in\RR : x^2\leq 2\}$, and note its upper-boundedness.
  Try to find its least upper bound. (hint: contradiction).
\end{proof}§
\begin{theorem}[Intermediate value theorem]
  Let $f:[a,b]\to\RR$ be continuous. Suppose $f(a)=c$ and $f(b)=d$ and $c<d$.
  For any real number $q\in(c,d)$, there exists some $p\in(a,b)$ s.t. $f(p)=q$.
  \label{thm:ivt}
\end{theorem}
\begin{proof}
  TODO.
\end{proof}
\subsubsection{Cauchy's inequality}
\begin{lemma}[Cauchy's inequality]
  Let $v=(v_1,\dots,v_n)$ and $w=(w_1,\dots,w_n)$ be vectors in $\RR^n$. We have 
  \[|v\cdot w|=|v||w| \]
  Where $|v|=\sqrt{v\cdot v}$, the length of $v$.
  \label{lem:cauchyIneq}
\end{lemma}
\begin{proof}
  Consider the quadratic function $f(\lambda)=|(v+\lambda w)|^2=w\cdot w\lambda^2
  + 2v\cdot w \lambda + v\cdot v$, which is positive for all $\lambda\in\RR$
  (it's just the square of the lenght of $(v+\lambda w)$), and so it must have
  at most one root. Hence the discriminant $b^{2}-4ac$ is non-positive, so 
\[(2v\cdot w)^2 - 4 w\cdot w v\cdot v \leq 0\]
\[\implies |v\cdot w| \leq \sqrt{w\cdot w} \sqrt{v\cdot v}\]
\end{proof}
