\section{Week 7 - 25 Feb 2022 - Continuous functions between topological spaces }
Recall that we have several characterisations of continuity of a function
$f:X\to Y$,
\begin{enumerate}
  \item For every $a\in X$ and $\eps>0$, there exists $\delta>0$ s.t. 
    \[d_X(a,x)<\delta \implies d_Y(f(a),f(x))<\eps.\]
  \item For every $a\in X$ and $\eps>0$ there exists $\delta>0$ s.t.
    \[f(B_X(a,\delta))\subset B_Y(f(a),\eps).\]
  \item For a convergent sequence $x_n\to a$ in $X$ we have $f(x_n)\to f(a)$ in
    $Y$.
  \item If $U$ is open $Y$ then $f^{-1}(U)$ is open in $X$.
\end{enumerate}
Note that $1,2$ depend explicitly on the metric, so they're not suitable for the
generalisation in topological spaces. $4$ is the natural choice, since it works
without explicit metric.
\begin{definition}
  Let $X$ and $Y$ be topological spaces. Then $f:X\to Y$ is continuous if for
  every open subset $U\subset Y$ we have the preimage $f^{-1}(U)\subset X$ is
  open in $X$.
  \label{def:topologicalContinuity}
\end{definition}
\begin{exercise}
  $f:X\to Y$ is continuous iff $V\subset Y$ closed implies $f^{-1}(V)\subset X$
  is closed.
\end{exercise}
By using Definition \ref{def:topologicalContinuity}, although different form
what we're used to, some familiar properties survive.

\begin{lemma}
  For any topological space $X$, the identity function $f:X\to X:x\mapsto x$ is
  continuous
  \label{<+label+>}
\end{lemma}
\begin{proof}
  TODO.
\end{proof}
\begin{lemma}
  If $X,Y,Z$ are topoological spaces and $f:X\to Y$ and $g:Y\to Z$ are
  continuous, then the composition $g\circ f:X\to Z$ is also continuous.
  \label{<+label+>}
\end{lemma}
\begin{proof}
  TODO.
\end{proof}

\begin{example}
  Let $X,Y$ be topological spaces, and $Y$ be the indescrete topology (where
  only $\emptyset$ and $Y$ are open subsets). Then every function $f:X\to Y$ is
  continuous. This is because $f^{-1}(\emptyset)=\emptyset$ (by definition of a
    function, every point in the domain must be mapped, hence no element is
  mapped to the emptyset), and $f^{-1}(Y)=X$, where $\emptyset, X$ are open in
  $X$ by the axioms of topological space. 
\end{example}

\subsection{Homeomorphisms}
\begin{definition}
  Let $(X,T_X)$ and $(Y,T_Y)$ be topological spaces. A function $f:X\to Y$ is
  called a \emph{homeomorphism} if $f$ is a bijection and $U\in T_X \iff
  f(U)\in T_Y$. We say $X,Y$ are \emph{homeomorphic} if there exists a
  homeomorphism $f:X\to Y$. We write $f:X\tohom Y$ to indicate a
  homeomorphism, and $X\cong Y$ to mean that $X,Y$ are homeomorphic.
  \label{def:homeomorphism}
\end{definition}
Two spaces $X,Y$ being homeomorphic by $f$ are considered to be the same since
the bijection gives a dictionary between elements of $X$ and elements of $Y$,
hence it translates naturally the topologies line up precisely.

\begin{proposition}
  The homoemorphism gives an equivalence relation between topological spaces.
  I.e. if $X,Y,Z$ are topological spaces, 
  \begin{enumerate}
    \item $X\cong X$,
    \item $X\cong Y \iff Y\cong X$,
    \item $X\cong Y \land Y\cong Z \implies X\cong Z$.
  \end{enumerate}
  \label{<+label+>}
\end{proposition}
\begin{proof}
  TODO. For (1) prove that the identity map is a homeomorphism, for (2) prove
  that a homeomorphism's inverse is also a homeomorphism, and (3) the
  composition of homeomorphisms is a homeomorphism.
\end{proof}

\begin{proposition}
  Let $X,Y$ be topological spaces. A function $f:X\to Y$ is a homeomorphism iff
  \begin{enumerate}
    \item $f$ is bijective
    \item $f$ is continuous,
    \item $f^{-1}$ is continuous.
  \end{enumerate}
  \label{<+label+>}
\end{proposition}
\begin{proof}
  Since a homeomorphism is bijective, we're only left to check 2,3 being
  equivalent to $U\in T_X \iff f(U)\in T_Y$ for a bijection $f$. Note that when
  $f$ is bijective, for every $U\subset X, V\subset Y$, we have 
  \[ U=f^{-1}f(U), V=f(f^{-1}(V)), f(U)= (f^{-1})^{-1}(U),\]
  By using injectivity, surjectivity, and inverse existence, respectively. 
  Then by continuity, note that $f(U)$ is open in $Y$ (so $f(U)\in T_Y$) implies
  $f^{-1}(f(U))=U$ is open in $X$ (so $X\in T_X$). Moreover, by continuity of the
  inverse $f^{-1}$ we have $U$ is open in $X$ (so $U\in T_X$) implies
  $f(U)=(f^{-1})^{-1}(U)$ is open in $Y$ (so $f(U)\in T_Y$). Hence (2,3) implies
  Definition \ref{def:homeomorphism}.

  Conversely, by the defintion of homeomorphism, we have 
  \[V=f(f^{-1}(V)) \text{ open in } Y \implies f^{-1}(V) \text{ open in } X,\]
  Hence $f$ is continuous. Moreover,
  \[U \text{ open in } X \implies (f^{-1})^{-1}(U)=f(U) \text{ open in } Y\]
\end{proof}
\begin{remark}
  We will eventually see examples of functions whose inverse exist but are not
  continuous, hence that function is not a homemorphism. Recall from group
  theory that any bijective group homomorphism is an isomorphism of groups.
  However, in topological spaces, a bijective map is not enough: it needs to be
  continuous and its inverse too.
  \label{<+label+>}
\end{remark}

\begin{proposition}
  Any two bounded open intervals in $\RR$ with the Euclidean topology are
  homeomorphic.
  \label{<+label+>}
\end{proposition}
\begin{proof}
  Note that for any $a<b$ the map $f:(0,1)\to (a,b):x\mapsto a+x(b-a)$ is a
  homeomorphism (TODO). Thus any open interval in $\RR$ is homeomorphic to
  $(0,1)$, and between themselves by transitivity.
\end{proof}
\begin{remark}
  The Euclidean topology is the topology induced by the Euclidean metric (hence
  open subsets in this topology are open under the Euclidean metric).
  \label{<+label+>}
\end{remark}
