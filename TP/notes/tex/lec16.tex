\section{Week 7 - 23 Feb 2022 - Intro to topological spaces}
Note that there're not Week 6 notes because of strikes.
\begin{definition}
  Let $X$ be a set. A topology on $X$ is a collection $T$ of subsets of $X$,
  called \emph{open sets}, satisfying three axioms,
  \begin{enumerate}
    \item The whole set and empty set are open, i.e. $X,\emptyset\in T$,
    \item Arbitrary unions of open sets are open, i.e. for $U_i\in T$ for all
      $i\in I$ we have $\cup_{i\in I} U_i\in T$ ($I$ is some indexing set,
      finite or not),
    \item Finite intersections of open sets are open, i.e. $U_1,\cdots, U_n\in T
      \implies \cap_{i=1}^n U_i \in T$.
  \end{enumerate}
  A topological space $(X,T)$ is a set $X$ with a topology $T$ on $X$.
  \label{def:topology}
\end{definition}
\begin{remark}
  Like with metric spaces, we often say $X$ is a topological space, if it's
  clear what topology we're using. On another hand, we usually say $U$ is open
  in $X$, rather than saying $U\in T$.
\end{remark}
\begin{remark}
  The third axiom for a topology is equivalent to saying that for any $U,V$ open
  subsets of $X$, then $U\cap V$ is open in $X$. The axiom then follows by
  induction.
\end{remark}
\begin{example}
  Let $(X,d)$ be a metric space, and let $T_d$ be the collection of subsets of
  $X$ open wrt $d$. Then $(X,T_d)$ is a topological space. Hence we get a
  function $f:(X,d)\mapsto (X,T_d)$ from the set of metric spaces to the set of
  topological spaces. This is 
  \begin{enumerate}
    \item Not injective: Topologically equivalent metrics induce the same
      topology. For example, if $X=\RR^n$ then $T_{d_1}=T_{d_2}=T_{d_{\infty}}$,
    \item Not surjective: We'll talk more about this later.
  \end{enumerate}
\end{example}
\begin{definition}
  A topological space $(X,T)$ is metrizable if $T=T_d$ for some metric $d$ on
  $X$
  \label{<+label+>}
\end{definition}
\begin{definition}
  Let $X$ be a set. The discrete topology on $X$ is the topology in which every
  subset of $X$ is open.
  \label{<+label+>}
\end{definition}
\begin{exercise}
  Show that the discrete topology on $X$ is metrizable.
\end{exercise}
\begin{proof}[Solution]
  TODO. Hint: Show that it is a $T_d$ for $d$ the discrete metric.
\end{proof}
\begin{proposition}
  Let $X$ be afinite set and $d$ any metric on $X$. Then $T_d$ is the discrete
  topology.
  \label{<+label+>}
\end{proposition}
\begin{proof}
  Recall that any finite subset of a metric space is closed. Thus, since $X$ is
  finite, all its subsets are closed, and so by taking complements, all its
  subsets are open.
\end{proof}
\begin{remark}
  Warning, a subset being open does not imply that the subset is closed.
  \label{<+label+>}
\end{remark}

\begin{corollary}
  On a finite set, the only metrizable topology is the discrete topology.
  \label{<+label+>}
\end{corollary}
The discrete topology has as many open sets as possible.  At the other extreme,
we have the following.
\begin{definition}
  Let $X$ be any set. The indiscrete topology on $X$ is the topology in which
  the only open sets are $X$ and $\emptyset$.
  \label{<+label+>}
\end{definition}
\begin{example}
  Let $X=\{a,b\}$. The subsets of $X$ are $\emptyset, \{a\},\{b\},X$. Hence
  there are $2^4=16$ possible collections of subsets. Any topology on $X$ must
  include $\emptyset$ and $X$, leaving four possibilities: 
  \[I_1=\{\emptyset, X\},\]
  \[I_2=\{\emptyset, X, \{a\} \},\]
  \[I_3=\{\emptyset, X, \{b\} \},\]
  \[I_4=\{\emptyset, X,\{a\}, \{b\} \},\]
  Where $I_1$ is the indiscrete and $I_4$ is the discrete topology. Observe that
  $I_1,I_2,I_3$ because $X$ is a finite set and they are not discrete. Note
  $I_2,I_3$ differ only by swapping $a$ by $b$, an example of a homeomorphism
  (think of isomorphism) between $(X,T_2)$ and $(X,T_3)$. On the other hand,
  $I_1,I_4$ are fundamentally different from $I_2,I_3$ and from each other,
  since they have different cardinality.
\end{example}
Our next aim is to generalise the theory we developed for metric spaces to
general topological spaces. We start with closed sets and then, most
importantly, will define continuous functions between topological spaces.

\subsection{Closed sets in topological spaces}
\begin{definition}
  Let $X$ be a topological space. A subset $V\subseteq X$ is closed if its complement
  $X\setminus V$ is open (i.e. $X\setminus V\in T$).
  \label{<+label+>}
\end{definition}
We may restate the axioms of a topological space in terms of closed sets using
de Morgan's laws.
\begin{proposition}
  Let $X$ be a set, and let $V$ be a collection of subsets of $X$. Then $V$ is
  the collection of closed sets for some topology $T$ on $X$ (necessarily
  $T=\{X\setminus v: v\in V\}$) if and only if
  \begin{enumerate}
    \item $X,\emptyset \in V$,
    \item $v_i\in V$ for all $i\in I \implies \cap v_i\in V$
    \item $v_1,\dots, v_n\in V \implies \cup_{i=1}^n v_i\in V$.
  \end{enumerate}
  \label{<+label+>}
\end{proposition}
\begin{proof}
  Exercise. Hint: Use de Morgan's laws.
\end{proof}

Thus we can specify a topology by saying what its closed sets are, checking that these
sets satisfy conditions 1-3 of this proposition.

\begin{definition}
  Let $X$ be a set. The cofinite topology on $X$ is the topology in which the
  closed sets are $X$ itself and the finite subsets of $X$. 
  \label{<+label+>}
\end{definition}
 Let's check that this is a topology: 
 \begin{enumerate}
   \item $X$ is closed by definition, and $\emptyset$ is closed (it is finite),
   \item Let $v_i,i\in I$ be closed. Then either $v_i =X$ for all $i\in I$, so
     finite intersections is just $X$, closed, or $v_i$ is finite for some $i\in
     I$, so finite interesctions is also closed.
   \item Let $v_1,.., v_n$ be closed. Then either $v_j=X$ for some $1\leq j\leq
     n$, so union for all $j$ is closed, or $v_i$ is finite for some $1\leq
     j\leq n$, so unions is also finite.
 \end{enumerate}
