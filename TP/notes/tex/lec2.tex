\section{Week 1 - 12 Jan 2022 - Open Balls}
\begin{example}[The French railroad metric]
  Take $X=\RR^2$ with $D(x,y)$ defined as follows
  \[D(x,y) = 
    \begin{cases}
      d_2(x,y), & x,y,0 \text{ are colinear} \\
      |x|+|y|, & \text{otherwise}
    \end{cases}
  \]
  Geometrically, it's rather intuitive. The name comes because in Paris
  railroad, if you're unlucky to not have your destination in the same line
  (e.g. trian line), then you'll have to go to paris first. 
\end{example}
To see how a metric behaves on a space, we consider the open balls.
\begin{definition}[Open Ball]
  Let $(X,d)$ be a metric space, and let $x\in X$ and $r>0$ in $\RR$. The open
  ball of radius $r$ about $x$ is the set
  \[B_{X}(x,r)=\left\{ y\in X : d(x,y)<r \right\}\]
  \label{def:openBall}
\end{definition}
\begin{remark}
  We should write $B_{(X,d)}$, since the ball also depends on the metric, but we
  use the abbreviation.
\end{remark}
\begin{example}
  Recall $d_1(x,y)=|x_1-y_1|+|x_2-y_2|$,
  $d_2(x,y)=\sqrt{(x_1-y_1)^2+(x_2-y_2)^2}$, $d_{\infty}=\max\left\{
  |x_1-y_1|, |x_2-y_2| \right\}$, and $D$ from above, are all metrics on
  $\RR^2$. However, their open balls centered at $0$ look rather different for
  each. However, for $D$, it looks exactly like $d_2$. Interestingly enough,
  its open ball centered at $(1,1)$, the open ball for $D$ looks more
  interesting: an open interval line-looking through $(1,1)$ not including the
  origin (only a line, since in the non-colinear direction, the metric $D$ gives
  values $>1$ !).
\end{example}

\begin{lemma}
  For any $x,y\in\RR^2$, the following inequalities hold,
  \begin{enumerate}
    \item $d_{\infty}(x,y) \leq d_{2}(x,y) \leq d_1(x,y)$
    \item $d_1(x,y) \leq \sqrt{2}d_2(x,y) \leq 2 d_{\infty}(x,y)$
  \end{enumerate}
\end{lemma}
\begin{proof}
  Without loss of generality (WLOG), we assume that $\max\left\{
  |x_1-y_1|,|x_2-y_2| \right\} = |x_1-y_1|$. This implies that $d_{\infty}(x,y)
  = |x_1-y_1| = \sqrt{(x_1-y_1)^2} \leq \sqrt{(x_1-y_1)^2 + (x_2-y_2)^2} =
  d_2(x,y)$, hence $d_{\infty}(x,y)\leq d_{2}(x,y)$. On the other hand, we claim
  $\sqrt{(x_1-y_2)^2 + (x_2-y_2)^2} \leq |x_1-y_1| + |x_2-y_2|$. Observe that 
  \[\iff (x_1-y_2)^2 + (x_2-y_2)^2 \leq |x_1-y_1|^2 + |x_2-y_2|^2 + 2|x_1-y_1||x_2-y_2|\]
  \[\iff 0\leq 2|x_1-y_1||x_2-y_2|\]
  This is certainly true. Hence the result for (1) follows.

  For (2), we observe that $|x_1-y_1| + |x_2-y_2| \leq \sqrt{2}\sqrt{(x_1-y_2)^2 +
  (x_2-y_2)^2}$,
  \[\iff |x_1-y_1|^2 + |x_2-y_2|^2 + 2|x_1-y_1||x_2-y_2| \leq 2(x_1-y_2)^2 + 2(x_2-y_2)^2\]
  \[\iff 2|x_1-y_1||x_2-y_2| \leq (x_1-y_2)^2 + (x_2-y_2)^2\]
  \[\iff 0\leq(x_1-y_2)^2 + (x_2-y_2)^2 - 2|x_1-y_1||x_2-y_2| \]
  \[\iff 0 \leq (|x_1-y_1| - |x_2-y_2|)^2\]
  Which is certainly true. On the other hand we have $\sqrt{2}d_2(x,y)\leq
 2d_{\infty}(x,y)$ (recall we have $\max\left\{
 |x_1-y_1|,|x_2-y_2| \right\} = |x_1-y_1|$),
 \[\iff 2(x_1-y_1)^2 + 2(x_2-y_2)^2 \leq 4(x_1-y_1) \iff 2(x_2-y_2)^2 \leq
 2(x_1-y_1)^2 \]
 Which is certainly true. Hence the result follows.
\end{proof}

\begin{definition}[Strong equivalence of metrics]
  Let $d,d'$ be metrics on a set $X$. We say that $d$ and $d'$ are
  \emph{strongly equivalent} if there exists real constants $c_1,c_2\in\RR$ s.t.
  \[d(x,y)\leq c_1 d'(x,y) \land d'(x,y) \leq c_2 d(x,y)\]
  for all $x,y\in X$. It is also called \emph{Lipschitz equivalence}.
\end{definition}
We will eventually see (formally) that the important propoerties of a metric
space only depend on the metric up to strong equivalence.

\begin{theorem}
  The metrics $d_1,d_2,d_{\infty}$ are all strongly equivalent on $\RR^n$ for
  any $n\geq 1$.
  \label{thm:d1d2d8StrongEquivalence}
\end{theorem}
\begin{proof}
  Exercise Sheet 1, Challenge question 4.
\end{proof}
\begin{exercise}
  The strong equivalence of metrics is an equivalence relation.
\end{exercise}
\begin{proof}[Solution]
  We show reflexivity, symmetry, and transitivity. Let $d,d'$ be metrics, and
  let $d\sim d'$ denote strong equivalence. We have that $d(x,y)=d(x,y)$, hence
  reflexivity ($d\sim d$). Second, we have that $d(x,y)\leq c_1d'(x,y) \land d'(x,y)\leq
  c_2d(x,y)$, hence it follows that $\frac{1}{c_1}d(x,y)\leq d'(x,y) \land
  \frac{1}{c_2}d'(x,y)\leq d(x,y)$, therefore symmetry ($d\sim d' \iff d'\sim
  d$). Finally we have transitivity. Let $d''$ be another metric s.t. $d\sim
  d''$, and assume $d''\sim d'$ so that 
  \[d(x,y)\leq c_1 d''(x,y) \land d''(x,y) \leq c_2 d(x,y)\]
  \[d'(x,y)\leq c_3 d''(x,y) \land d''(x,y) \leq c_4 d'(x,y)\]
  Then we have 
  \[d(x,y)\leq c_1c_4 d'(x,y) \land d'(x,y) \leq c_2c_3d(x,y)\]
  Therefore we have $d\sim d'$, i.e. transitivity. Henceforth we have that
  $\sim$ is an equivalence relation.
\end{proof}
\begin{example}
  The French railroad metric $D$ is not strongly equivalent to any
  $d_1,d_2,d_{\infty}$, hence it's not strongly equivalent to any, since it's an
  equivalence relation. Assume $\exists c\in\RR$ s.t. $D(x,y)\leq cd_2(x,y)$ for
  all $x,y\in\RR^2$. Take $x_n=(n,0), y_n=(n,1)$ for $n\in\NN$. We have
  $d_2(x_n,y_n)=1$ but $D(x_n,y_n)=|x_n|+|y_n|=n^2+\sqrt{n^2+1}\geq n$. Hence
  $n\leq D(x_n,y_n) \leq cd_2(x_n,y_n)=c$ for all $n\in\NN$, therefore a
  contradiction.
\end{example}
