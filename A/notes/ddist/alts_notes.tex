%% 
%% This is file, `alts_notes.tex',
%% generated with the extract package.
%% 
%% Generated on :  2021/12/30,1:11
%% From source  :  notes.tex
%% Using options:  active,generate=alts_notes,extract-env={definition},extract-cmd={section,subsection}
%% 
\documentclass[11pt]{scrartcl}
\input{tex/preamble}
\title{Algebra}
\subtitle{Taught by Alex Bartel}
\author{Allan Perez}
\date{Fall 2021}

\begin{document}
 \maketitle

\tableofcontents
\newpage

\section{Lecture 1 - 21 Sep 2021}

\begin{definition}
  A group is a pair $(G,*)$ where $G$ is a set and $*:G\times G\to G$ is a binary
  operation with the following axioms:
  \begin{itemize}
      \ii (Associativity) For any $g,h,k\in G$
      \[(g*h)*k = g*(h*k)\]
      \ii (Existence of identity) There exists $e\in G$ such that for any $g\in G$,
      \[e*g=g*e = g\]
      \ii (Existence of inverse) For any $g\in G$ there exists $h\in G$ s.t.
      \[g*h=h*g=e\]
  \end{itemize}
  \label{group}
\end{definition}

\begin{definition}
  A group $G$ is abelian if the group is commutative, i.e. $\forall g,h\in G$ we have
  $gh=hg$.
  \label{abelianGroup}
\end{definition}

\begin{definition}
  We say a group is finite or countable if the underlying set is finite or countable,
  respectively.
\end{definition}

\subsection{Symmetric groups}

\section{Lecture 2 - 22 Sep 2021}

\begin{definition}
   A subgroup of a group G, $H\subset G$, is a subset of $G$ with
   \begin{itemize}
       \ii It contains the $G$ group identity element.
       \ii For all $a,b\in H$ we have $ab\in H$
       \ii for any $a\in H$ we have $a^{-1}\in H$.
   \end{itemize}
   \label{subgroup}
 
\end{definition}

\begin{definition}
  A group $G$ is called cyclic if $\exists g\in G$ such that $G=\left\{ g^n : n\in\ZZ
  \right\}$. If $G$ is cyclic, then an element $g$ as above is called a generator, and we
  say $G$ is generated by $g$, $G=<g>$.
  \label{cyclicGroup}
\end{definition}

\begin{definition}
  The order of an element $g\in G$ of a group $G$, written $|g|$, is the least positive
  integer $n$ s.t. $g^n=e$. If such $n$ doesn't exist, it has infinite order. The order
  of a group $G$, $|G|$, is the cardinallity of the underlying set.
  \label{orderGroup}
\end{definition}

\subsection{Cyclic groups structure}

\section{Lecture 3 - 28 Sep 2021}

\subsection{Multiplication table}

\section{Lecture 4 - 27 Sep 2021}

\subsection{Cosets}

\begin{definition}
  Let $G$ be a group and $H$ a subgroup of $G$. Let $g\in G$. The left coset of $H$
  containing $g$ is the set $gH=\left\{ gh | h\in H \right\}$. The right coset of $H$
  containing $g$ is the set $Hg=\left\{ hg | h\in H \right\}$.
  \label{coset}
\end{definition}

\section{Lecture 5 - 29 Sep 2021}

\subsection{Counting groups and Lagrange}

\begin{definition}
  Let $G$ be a group and $H$ a subgroup. The set of left cosets of $H$ in $G$ is denoted
  by $G/H$, also called the set of equivalence classes generated by the equivalence
  relation defined in Corollary \ref{leftCosetsEqRel} . Similarly the set of right cosets of
  $H$ in $G$ is denoted by $H\setminus G$.
  \label{cosets}
\end{definition}

\begin{definition}
  The number of (left) cosets of $H$ in $G$ is called the index of $H$ in $G$, written
  $[G:H]$.
\end{definition}

\section{Lecture 6 - 4 Oct 2021}

\subsection{Normal subgroups and Quotients}

\begin{definition}
  Let $G$ be a group. A subgroup $H$ of $G$ is called normal, written $H\triangleleft G$
  or $H\trianglelefteq G$, if $\forall g\in G$ we have $gHg^{-1}=H$.
  \label{normalSubgroup}
\end{definition}

\begin{definition}
  Let $G$ be a group, and $N$ be a normal subgroup. The set of left cosets $G/N$ together
  with the binary operation $(gN)(hN)=(gh)N$ for $g,h\in G$ is called the \emph{quotient
  group} or \emph{factor group} of $G$ by $N$.
  \label{quotientGroup}
\end{definition}

\section{Lecture 7 - 6 Oct 2021}

\begin{definition}
  Let $(H,*),(K,\cdot)$ be groups. We define their \emph{direct product} as the group
  with underlying set $H\times K = \left\{ (h,k) : h\in H, k\in K \right\}$ with operation
  of point-wise multiplication
  \[(h,k)(h',k')=(h*h',k\cdot k')\]
  \label{directProduct}
\end{definition}

\section{Lecture 8 - 8 Oct 2021}

\subsection{Group Homomorphisms, Types and Facts}

\begin{definition}
  Let $G,G'$ be groups. A group homomorphism from $G$ to $G'$ is a function $\phi:G\to G'$
  s.t. for all $g,h\in G$ one has $\phi(gh)=\phi(g) \phi(h)$.
  \label{groupHomomorphism}
\end{definition}

\begin{definition}[Morphism Zoo]
  \begin{enumerate}
    \item A group \emph{isomorphism} is a grou homomorphism $\phi:G\to G'$ that has a
      2-sided inverse. This is, $\phi\circ\phi^{-1}=1_{G'},\phi^{-1}\circ\phi=1_G$. If
      there exists a group isomorphism between groups $G,G'$, we say these groups are
      isomorphic and write $G\cong G'$. By the previous results (Theorems
      \ref{groupsCategories} \ref{homIdInv}) we have that being isomorphic is an
      equivalence relation. Two isomorphic groups are structurally the same.

    \item A group automorphism is a group isomorphism to itself.
    \item A group endomorphis is a group homomorphism to itself. E.g. the trivial
      homomorphism to itself is an endomorphism for non-trivial groups.
  \end{enumerate}
  \label{morphismZoo}
\end{definition}

\section{Lecture 9 - 11 Oct 2021}

\subsection{First Isomorphism Theorem}

\begin{definition} [Kernel of Homomorphisms]
  Let $\phi:G\to G'$ be a group homomorphism. Then the \emph{kernel} $\ker\phi$ is the set
  of all elements of $G$ that are mapped to the identity of $G'$,
  \[\ker \phi = \left\{ g\in G : \phi(g)=1_{G'} \right\}\]
  \label{kernel}
\end{definition}

\section{Lecture 10 - 11 Oct 2021}

\section{Lecture 11 - 13 Oct 2021}

\subsection{Second Isomorphism Theorem}

\subsection{Non-examinable aside}

\begin{definition}[Simple Group]
  A group is \emph{simple} if it is non-trivial and has no proper non-trivial normal
  subgroups.
\end{definition}

\section{Lecture 12 - 18 Oct 2021}

\subsection{Third Isomorphism Theorem}

\section{Lecture 13 - 20 Oct 2021}

\subsection{Group Actions}

\begin{definition} [Left action]
  Let $G$ be a group, and let $X$ be a set. A left action of $G$ on $X$ is a function
  \[G\times X \to X\]
  \[(g,x)\mapsto g\cdot x\in X\]
  Such that
  \begin{enumerate}
    \item $1_G \cdot x=x$ for any $x\in X$
    \item $g_1 \cdot (g_2\cdot x) = (g_1g_2)\cdot x$ for any $g_1,g_2\in G$ and $x\in X$.
  \end{enumerate}
  Note that $\cdot$ is used to denote the action and the multiplication notation used for
  group operation. A set equipped with an action of a group $G$ is called $G$-set.
  \label{def:leftAction}
\end{definition}

\begin{definition}[Right action]
  Analogously, a right action of a group $G$ on a set $X$ is a function
  \[X\times G \to X\]
  \[(x,g)\mapsto x\cdot g\]
  Such that
  \begin{enumerate}
    \item for all $x\in X$ one has $x\cdot 1_G=x$
    \item for all $g_1,g_2\in G$ and for all $x\in X$ one has $(x\cdot g_1)\cdot g_2 =
      x\cdot (g_1g_2)$.
  \end{enumerate}
  \label{def:rightAction}
\end{definition}

\begin{definition}
  Let $G$ be a group. A $G$-action on a set $X$ is called transitive if for any $x,y\in X$
  there exists $g\in G$ s.t. $y=g\cdot x$.
  \label{def:transitiveAction}
\end{definition}

\begin{definition} [Isomorphism on group action]
  Let $G$ be a group and $X,Y$ be $G$-sets. An isomorphism from $X$ to $Y$ is a bijection
  $\phi:X\to Y$ s.t. $\forall x\in X, g\in G$, we have $\phi(g\cdot x)=g\cdot \phi(x)$.
  \label{defi:isomorphismAction}
\end{definition}

\section{Lecture 14 - 22 Oct 2021}

\subsection{Orbit-Stabiliser Theorem}

\begin{definition}[Obrit and Stabiliser]
  Let $G$ be a group and $X$ be a $G$-set. Let $x\in X$. The orbit of $x$ in $G$ is
  $\Orb_G (x)= G\cdot x = \left\{ g\cdot x :g\in G\right\}$.
  The stabiliser of $x$ in $G$ is $\Stab_G (x)=\left\{ g\in G | g\cdot x=x \right\}$
  \label{def:orbStab}
\end{definition}

\section{Lecture 15 - 25 Oct 2021}

\subsection{Applications of Orbit-Stabiliser Theorem and Cauchy's Theorem}

\section{Lecture 16 - 27 Oct 2021}

\subsection{Not Burnside's Lemma}

\begin{definition}
  Let $G$ be a group and $X$ be a $G$-set. Let $g\in G$. Define
  \[X^g=\left\{ x\in X : g\cdot x =x \right\} \subset X\]
  The elements of $x$ that are fixed by $g$. Note that this definition is closely related
  to the stabiliser definition, but it's different.
\end{definition}

\begin{definition}
  Let $G$ be a group and $X$ be a $G$-set. The set of $G$-orbits of $X$ is denoted by
  $X/G$. Thus we have $X=\bigsqcup_{O\in X/G} O$ (note, disjoint union).
\end{definition}

\section{Lecture 17 - 29 Oct 2021}

\subsection{Semi-direct product}

\begin{definition}
  Let $G$ be a group and let $H\leq G$, $N\trianglelefteq G$ such that $H\cap N = \{e\}$
  and $HN=G$. Then $G$ is called an \emph{(internal) semi-direct product} of $H,N$.
  \label{def:intSemidirProd}
\end{definition}

\begin{definition}
  Given groups $N$ and $H$, and a group homomorphism $\phi:H\to \Aut N$ we define the
  \todo{How can we define this homomorphism if H and G are not even supposed to be
  subgroups of the same group? We defined before by conjugation but for conjugation to
make sense (hnh' in N) we need first for hn to be defined}
  \emph{(external) semidirect product} of $N$ and $H$ (wrt $\phi$), written $N\rtimes H$
  or $N\rtimes_{\phi} H$, with the multuplication operation,
  \[(n,h)(n',h')=(n\phi_h(n'), hh')\]
    \todo{Check associativity}
  \label{def:exSemidirProd}
\end{definition}

\section{Lecture 18 - 1 Nov 2021}

\subsection{Intro to rings}

\begin{definition}
  A ring is a collection $(R,+,\cdot)$ where $R$ is a set, and $+,\cdot$ are two binary
  operations on $R$ such that
  \begin{enumerate}
    \item $(R,+)$ is an abelian group (labeled as the additive identity $0$)
    \item The operation $\cdot$ is associative, i.e. for any $a,b,c\in R$ one has
      $(a\cdot b)\cdot c = a\cdot (b\cdot c)$
    \item $\cdot$ distributes over $+$, i.e. for all $a,b,c\in R$ one has
      \[(a+b)\cdot c = a\cdot c+b\cdot c \land c\cdot (a+b)=c\cdot a + c\cdot b\]
  \end{enumerate}
  A \emph{unital ring} or a \emph{ring with unity} is a ring as above such that there
  exists $1\in R\setminus \{0\}$ satisfying $1\cdot a = a\cdot 1 = a$ for all $a\in R$.
  A ring is called \emph{commutative} if the operation $\cdot$ is commutative. Note that
  $+$ is always commutative.
  \label{def:ring}
\end{definition}

\begin{definition}
  Let $R$ be a unital ring. An element $u$ of $R$ is called a \emph{unit} if there exists
  $u^{-1}\in R$ such that $uu^{-1}u^{-1}u=1$. The set of units of $R$ is denoted by
  $R^{\times}$. The set of units of $R$ forms a group under multiplication.
  \todo{Note that we defined $R$ to be closed under multiplication but it's not obvious
  that $R^{\times}$ is closed}
  \label{<+label+>}
\end{definition}

\begin{definition}
  A unital ring in which every non-zero element is a unit is called a \emph{division
  ring}. A commutative division ring is called a \emph{field}.
  \label{<+label+>}
\end{definition}

\section{Lecture 19 - 3 Nov 2021}

\subsection{Subrings, ideals, and quotients}

\begin{definition}
  Let $R$ be a ring. A subring is an additive subgroup $S\subset R$ s.t. for every
  $a,b\in S$, $ab\in S$. We usually write $S\leq R$. If $R$ is unital, a subring $S\leq R$
  is called a unital subring if $1\in S$. If $R$ is unital, then we assume every subring
  to be unital, unless otherwise stated.
  \label{<+label+>}
\end{definition}

\begin{definition}
  Let $R$ be a ring. A \emph{left ideal} of $R$ is an additive subgroup $I$ of $R$ s.t.
  for every $a\in I$ and $r\in R$, $ra\in I$. That is, for every $r\in R$, we have
  $rI\subseteq I$.

  A \emph{right ideal} of $R$ is an additive subgroup $I$ of $R$ s.t. for every $a\in I$
  and $r\in R$, $ar\in I$. That is, for every $r\in R$, we have $rI\subseteq I$.

  A \emph{two-sided ideal} is an additive subgroup $I$ of $R$ s.t. it's a left ideal and a
  right ideal. We write $I\trianglelefteq R$.

  An ideal of $R$ is called \emph{proper} if it's not equal to $R$.
  \label{<+label+>}
\end{definition}

\begin{definition}
  Let $R$ be a unital ring, and $I$ be a proper two-sided ideal in $R$. The
  \emph{quotient ring} has, as its underlying set, the set of cosets $\left\{ r+I : r\in R
  \right\}$. We define the addition of cosets as $(r+I)+(s+I)=(r+s)+I$ and multiplication
  as $(r+I)(s+I)=(rs)+I$.
  \todo{Think about how two-sided proper ideals implies that the operations are
  well-defined, i.e. do not depend on quotient representatives, where two elements
$r+I,s+I$ are said to be the same if $s=r+x \exists x\in I$.}
  \label{<+label+>}
\end{definition}

\section{Lecture 20 - 5 Nov 2021}

\subsection{Ring homomorphisms}

\begin{definition}
  Let $R,S$ be rings. Let a ring homomorphism from $R$ to $S$ be a function $\phi:R\to S$
  s.t. for every $a,b\in R$ we have
  \begin{enumerate}
    \item $\phi(a+b)=\phi(a)+\phi(b)$ (note that this defines a group homomorphism under
      the operation of addition).
    \item $\phi(ab)=\phi(a)\phi(b)$. (note no necessary a group homomorphism bc $R$ is not
      a multiplicative groups, but it must respect the operation).
  \end{enumerate}
  If $R,S$ are unital, then a ring homomorphism is called \emph{unital} if
  $\phi(1_R)=1_S$. Unless otherwise stated, a homomorphism between unital rings will be
  assumed to be unital.

  We say that a homomorphism is an isomorphism if it has a two-sided inverse
  $\phi\inv:S\to R$ s.t. it's also a ring homomorphism.
  \[\phi \circ \phi\inv = \id_S \quad \phi\inv\circ\phi = \id_R\]
  \label{def:ringHomUnital}
\end{definition}

\section{Lecture 21 - 8 Nov 2021}

\subsection{Isomorphism of rings and Cancellation}

\begin{definition}
  Let $\phi:R\to S$ be a ring homomorphism. The \emph{kernel} of $\phi$ is defined as
  $\ker\phi=\left\{ r\in R : \phi(r)=0_S \right\}$ and the \emph{image} of $\phi$ is
  defined as $\img\phi=\left\{ \phi(r) : r\in R \right\}\subset S$
\end{definition}

\begin{definition}
  Let $R$ be a ring. An element $a\in R$ is a \emph{left zero divisor} if $a\neq 0$ and
  there exists $b\in R\setminus \{0\}$ s.t. $ab=0$. A right zero divisor is defined
  analogously.
\end{definition}

\section{Lecture 22 - 10 Nov 2021}

\subsection{Integral domains}

\begin{definition}
  A commutative unital ring with no zero divisors is called an \emph{integral domain}.
  \label{def:integralDomain}
\end{definition}

\begin{definition}[Maximal ideal]
  Let $R$ be a ring. An ideal $I$ of $R$ is called \emph{maximal ideal} if $I$ is a proper
  ideal and for every other ideal $J$ s.t. $I\subseteq J \subseteq R$ we have that $J=I$
  or $J=R$.
  \label{def:maximalIdeal}
\end{definition}

\begin{definition}
  Let $R$ be a ring. An ideal $I$ of $R$ is called \emph{prime} if $I$ is a proper ideal,
  and whenever $a,b\in R$ are s.t. $ab\in I$, then one has $a\in I$ or $b\in I$.
  \label{<+label+>}
\end{definition}

\section{Lecture 23 - 12 Nov 2021}

\subsection{Prime and maximal ideals}

\begin{definition}
  Let $R$ be a ring, and $r\in R$. The ideal $Rr=\{xr | x\in R\}$ is called the \emph{left
  ideal generated by $r$}, and this is the smallest left sided ideal containing $r$. The
  \emph{two sided ideal generated by $r$} is
  \[RrR= \{\text{finite sums }\sum_i x_i r y_i : x_i,y_i\in R\}\]
  This is denoted by $(r)$ and is the smallest two-sided ideal containing $r$. An ideal of
  this form is called \emph{a principal ideal.} Moreover, if $S$ is a subset of $R$, then
  the ideal genereted by $S$ is
  \[RSR = \{\text{finite sums }\sum_{s\in S} x_s s y_s : x_s,y_s\in R \}\]

  \label{<+label+>}
\end{definition}

\section{Lecture 24 - 15 Nov 2021}

\subsection{Prime and maximal ideals - division with reminder in polynomials}

\begin{definition}
  Let $F$ be a field, and let $f(X)=a_0+a_1x+\cdots + a_d X^d\in F[X]$ be a polynomial
  with $a_d\neq 0$. The degree $\deg f$ of $f$ is defined as $d$.
  \label{<+label+>}
\end{definition}

\section{Lecture 25 - 17 Nov 2021 }

\subsection{Prime ideals in polynomial rings}

\begin{definition}
  Let $f\in F[X]$ be a polynomial over the field $F$ with $\deg f>0$. Then $f$ is
  \emph{irreducible} if whenever $g,h\in F[X]$ are such that $f=gh$, one has either $\deg
  g=0$ or $\deg h=0$. If $f$ is not irreducible, then it's called \emph{reducible}.
\end{definition}

\section{Lecture 26 - 19 Nov 2021}

\subsection{Irreducible polynomials}

\section{Lecture 27 - 22 Nov 2021}

\subsection{Intermission: Classifying groups of order 21}

\section{Lecture 28 - 24 Nov 2021}

\subsection{Irreducibility criteria}

\begin{definition}
  A polynomial $f(X)$ ins $\ZZ[X]$ is called primitive if the $\gcd(a_n,\cdots,a_0)=1$,
  i.e. if there exists no prime number $p$ that divides all $a_i$.
  \label{def:primitivePol}
\end{definition}

\section{Lecture 29 - 26 Nov 2021}

\subsection{Ring theory -- Fixing loose ends}

\section{Lecture 30 - 29 Nov 2021}

\subsection{Field Extensions (Non-examinable from now on)}

\begin{definition}[Field Extension]
  Suppose $K$ is a field, and let $F$ be a field containing $K$. Then $F/K$ is a
  \emph{field extension}.
  \label{def:fieldExt}
\end{definition}

\begin{definition}
  Let $F/K$ be a field extension, and $\alpha\in F$. We say $\alpha$ is algebraic over
  $K$ if there exists $f\in K[X]\setminus \{0\}$ s.t. $f(\alpha)=0$. If there's not such
  polynomial, we say that $\alpha$ is trascendental over $K$.
\end{definition}

\begin{definition}
  A polynomial $f=a_nx^n+\cdots+a_0\in K[X]$ is called \emph{monic} if $a_n=1$.
  Let $F/K$ be an extension field and $\alpha$ be algebraic over $K$. The unique monic
  irreducible polynomial $p\in K[X]$ s.t. $p(\alpha)=0$ is called the irreducible
  polynomial of $\alpha$ over $K$, or the minimal polynomial of $\alpha$ over $K$, written
  as $\irr (\alpha, K)$.
\end{definition}

\section{Lecture 31 - 01 Dec 2021}

\subsection{Field Extensions and Degrees}

\begin{definition}
  Let $F/K$ be a field extension and $\alpha\in F$ be algebraic over $K$. The
  \emph{degree of $\alpha$ over $K$}, written $\deg(\alpha, K)$, is the degree of the
  polynomial $\irr(\alpha, K)$.
  The \emph{degree of $F$ over $K$ }, written $[F:K]$ is the dimension of $F$ as a vector
  space over $K$. An extension $F/K$ is finite if $[F:K]<\infty$.
\end{definition}

\begin{definition}
  Let $K(\alpha)$ denote the smallest subfield of $F$ that contain $K$ and $\alpha$, the
  field generated by $\alpha$ over $K$
\end{definition}

\section{Lecture 32 - 03 Dec 2021}

\subsection{Algebraic Closure}

\begin{definition}
  A field $F$ is called algebraically closed if every $f\in F[X]$ with $\deg f>0$ has a
  root in $F$.
  \label{<+label+>}
\end{definition}

\end{document}
