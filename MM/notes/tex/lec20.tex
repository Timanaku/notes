\section{Lecture 20 - 3 Nov 2021}
\begin{example}
  Solve the BVP $L[y]=y''=f(x)$ with BCs $y(0)=y(\pi)=0$.
\end{example}
\begin{proof}
  The associated eigenvalue problem is 
  \[y''=-\lambda\omega(x)y\]
  Which we've solved before for $x\in[0,L]$. Let $L=\pi$, so the solution yields
  \[\lambda_n=n^2\]
  With eigenfunctions
  \[\phi_n = A_n\sin(nx)\]
  Normalising the eigenfunction, ($\langle \phi_n,\phi_n \rangle =\int \phi_n\phi_n
  \omega(x)dx=1$):
  \[1=\int_{0}^{\pi} A_n^2 \sin^2(nx) dx = A_n^2\frac{\pi}{2} \]
  So $A_n= \sqrt{\frac{2}{\pi}}$. We then write the solution as a sum of these functions
  \[y(x)=\sum_{n=0}^{\infty} \alpha_n \sqrt{\frac{2}{\pi}}\sin(nx)\]
  So substituting this back into our BVP,
  \[L[y]=f(x) \implies f(x)= -\sum_{n=0}^{\infty} \alpha_n\lambda_n\phi_n \]
  So we multiply both sides by $\phi_m$ for some $m\in\NN$, and integrate,
  \[\int_{0}^{\pi} f(x)\phi_m dx = -\lambda_m \alpha_m\]
  Given orthonormality of $\phi_n$. Since $\lambda_m=m^2$, it follows that the solution is
  \[\alpha_m= -\frac{1}{m^2} \int_{0}^{\pi} f(x) \sqrt{\frac{2}{\pi}}\sin(mx) dx\]
  For the case $f(x)=x$, the solution follows
  \[\alpha_m= \sqrt{\frac{2}{\pi}} \left[ \frac{\pi}{m^3} (-1)^m \right]\]
  The integration is left as an exercise to the reader (cos no one cares, it's just an
  integration by parts). Hence we find the solution for the case $f(x)=x$ to be 
  \[y(x)= \sum_{n=1}^{\infty} \frac{2(-1)^n}{n^3} \sin(nx)\]
  Which is a Fourier series.
\end{proof}

\subsection{Green's Functions}
Green's functions can be used to solve a wide family of PDEs, BVPs, and IVPs.
\emph{Good} functions are functions that live in $\SCC^{\infty}$ and decay rapidly, so
that the derivatives decay to $0$ faster than $|x|^{-n}$ as $x\to\infty$ for any $n>0$. An
example of such a function is $e^{-x^2}$.

We then can talk about generalised functions, of which the following are examples
\begin{enumerate}
  \item Heaviside $H$-function: For some \emph{good} function $F(x)$, we define $H$ as
    $\int_{-\infty}^{\infty} H(x)F(x)dx = \int_{0}^{\infty} F(x) dx$.
  \item Sign $\sgn$-function: For some \emph{good} function $F(x)$, we define $\sgn$ as
    $\int_{-\infty}^{\infty} \sgn(x)F(x) dx = \int_{0}^{\infty}F(x)dx - \int_{-\infty}^0
    F(x)dx$.
  \item Dirac $\delta$-function: For some good function $F$, we define $\delta$ as
    $\int_{-\infty}^{\infty} \delta(x-x_0) F(x) dx = F(x_0)$. This delta function can be
    taken to be the limit of a sequence of normal distributions,
    $\delta_n=\frac{1}{n\sqrt{\pi}}e^{-x^2/n}$, so $\delta=\lim_{n\to\infty}\delta_n$,
    which when integrating over $\RR$ gives $1$.
\end{enumerate}

\subsection{Green's function method}
Consider $L[y]=ay''+by'+cy$ and the equation $L[y]=f(x)$ with
$a,b,c\in\SCC[\alpha,\beta]$.§
\begin{definition}[Green's Function]
  The Green's function $G(x, \xi)$ of $L$ is the unique solution to the problem
  \[L[G]= \delta(x-\xi)\]
  With the BCs $G(\alpha,\xi)=G(\beta,\xi)=0$.
  \label{def:greenFun}
\end{definition}

