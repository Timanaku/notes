\section{Lecture 17 - 27 Oct 2021}
\begin{definition}
  A linear operator $L$ is adjoint on $\SCC[a,b]$ if
  \[\langle u,L[v] \rangle =\langle \hat{L}[u],v \rangle \]
  For all $u,v\in\SCC[a,b]$, where $\hat{L}$ is the adjoint operator of $L$.
  \label{<+label+>}
\end{definition}
\begin{definition}
  A linear operator $\hat{L}$ is self-adjoint on $\SCC[a,b]$ if 
  \[\langle u,L[v] \rangle =\langle L[u], v \rangle \]
  For all $u,v\in\SCC[a,b]$.
  \label{<+label+>}
\end{definition}

\begin{example}
  Find a condition for the Sturm-Liouville operator
  \[\hat{L}=\frac{d}{dx}(p(x)\frac{d}{dx}) + q(x)\]
  To be self-adjoint with respect to the inner product
  \[\langle u,v \rangle =\int_a^b u(x)v(x)dx \]
\end{example}
\begin{proof}[Solution]
  We have 
  \[\langle u,L[v] \rangle -\langle L[u],v \rangle = \int_a^b uL[v]-L[u]v dx\]
  Using Green's identity we have 
  \[=\left[ p(x)(uv'-u'v) \right]_a^b =0\]
  For this to be true, we require,
  \[ p(b)(u(b)v'(b)-u'(b)v(b))- p(a)(u(a)v'(a)-u'(a)v(a))=0\]
  Which is entirely set by boundary conditions. Hence the Sturm-Liouville operator is self
  adjoint wrt $\langle  \rangle $ defined above depending on its initial conditions.

  Recall that the \emph{unmixed boundary conditions} is a system of homogeneous equation which may be
  written in matrix form as 
  \[ A \alpha =
    \begin{bmatrix}
      u(a) & u'(a) \\
      v(a) & v'(a) 
    \end{bmatrix}
    \begin{bmatrix}
      \alpha_1\\
      \alpha_2\\
    \end{bmatrix} = 
    \begin{bmatrix}
      0\\0
    \end{bmatrix}
  \]
  Which will have nontrivial solutions if and only if $\det A=0$. For the same matrix we
  define the other boundary condition $\beta$ 
  \[ A\beta=
    \begin{bmatrix}
      u(a) & u'(a) \\
      v(a) & v'(a) 
    \end{bmatrix}
    \begin{bmatrix}
      \beta_1\\
      \beta_2\\
    \end{bmatrix} =  0
  \]
  These two boundary conditions give 
  \[=\left[ p(x)(uv'-u'v) \right]_a^b =0\]
  Hence the $\hat{L}$ is self-adjoint wrt $\langle  \rangle$ for unmixed boundary
  conditions.

  For \emph{periodic BCs} if $p(a)=p(b)\neq 0$ the operator is also guaranteed to be
  self-adjoint.
\end{proof}

\subsection{Orthogonality}
\begin{remark}
  We define the above inner product for a vector space over $\CC$ as 
  \[\langle u,v \rangle =\int_a^b \overline{u(x)} v(x) dx\]
  Which yields the identities
  \[\overline{\langle u,v \rangle }=\langle v,u \rangle \]
  \[\langle cf,g \rangle =\overline{c}\langle f,g \rangle \]
\end{remark}
\begin{theorem}
  The eigenvalues of a Sturm-Liouville eigenvalue problem are real.
  \label{<+label+>}
\end{theorem}
\begin{proof}
  Let $\phi_n$ be a solution of the eigenvalue problem associated with $\lambda_n$,
  \[\hat{L}(\phi_n)=-\lambda_n\omega\phi_n\]
  Where $\hat{L}$ is some Sturm-Liouville operator with functions $p,q$. The complex
  conjugate of the equation is
  \[\hat{L}(\overline{\phi_n})=-\overline{\lambda_n}\omega\overline{\phi_n}\]
  We can then operatre on the above equations and get
  \[\phi_n\hat{L}[\overline{\phi_n}]-\overline{\phi_n}\hat{L}[\phi_n]=(\overline{\lambda_n}-\lambda_n)w
  \phi_n\overline{\phi_n}\]
  By integrating both sides and using Green's identity we find, given unmixed boundary
  conditions,
  \[0=[p(x)(\phi_n'\overline{\phi_n})- \phi_n\overline{\phi_n}']_a^b =
    (\overline{\lambda_n}-\lambda_n)\int_a^b w\phi_n\overline{\phi_n} dx\]
    Since the integral is non-negative we must have
    $\overline{\lambda_n}=\lambda_n$, hence $\lambda_n\in\RR$.
\end{proof}
