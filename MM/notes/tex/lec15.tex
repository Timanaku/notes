\section{Lecture 15 - 22 Oct 2021}
\subsection{Boundary Value Problems and Sturm-Liouville Theory}
Boundary value problems (BVPs) are problems where conditions are given on the boundary or
extremes of independent variables in the equation, say $y(x=a),y(x=b)$ for $a<b$. Meanwhile, initial value
problems (IVP) have to specify two conditions at one point, $y(x=a), y'(x=a)$.
\begin{definition}
  A two point boundary value problem is given by a second order ODE
  $L[y]=a(x)y''+b(x)y'+c(x)y=f(x)$ subject to the unmixed homogeneous boundary conditions
  \[\alpha_1 y(a) + \alpha_2 y'(a)=0\]
  \[\beta_1 y(b) + \beta_2 y'(b)=0\]
  For $\alpha_i,\beta_i\in\RR$ and $a(x),b(x),c(x),f(x),y(x)$ being functions $\SCC [a,b]$.
  \label{<+label+>}
\end{definition}

Unmixed above means that one equation depends only on $a$ and the other only on $b$.
$\SCC [a,b]$ is the set of continuous functions on $[a,b]$ and $\SCC^2 [a,b]$ is the set
of twice-differentiable functions on $[a,b]$.

There're different boundary conditions:
\begin{enumerate}
  \item Newman boundary conditions: $y'(a)=y'(b)=0$
  \item Dirichlet boundary conditions: $y(a)=y(b)=0$
  \item Periodic boundary condition: $y(a)=y(b), y'(a)=y'(b)$ (Not homogeneous).
\end{enumerate}

We can solve some equations using eigenvalue expansions.
\begin{definition}
  The eigenvalues $\lambda$ and eigenfunction $\phi$ of the operator $L$ are non-trivial
  solutions to 
  \[L\phi = \lambda\phi\]
  Subject to appropriate boundary conditions.
  \label{def:eigenFun}
\end{definition}

If we can ding $\lambda_i$ and $\phi_i$ then we seek solutions to the BVP as an expansion
of the eigenfunctions,
\[y=\sum_{n=1}^{\infty} c_n\phi_n\]
Notice how it resembles power series expansion method. However, we're not guaranteed to
get nice eigenfunctions. It would be nice to have orthogonal eigenfunctions, this will
allow us to solve for $c_n$. Turns out self-adjoint operators have nice properties.
\begin{definition}[Sturm-Liouville operator]
  Let $L$ be an operator of the form
  \[L= \frac{d}{dx}(p(x)\frac{d}{dx}) + q(x)\]
  Notice $L=\hat{L}$, a self-adjoint operator. Then $L$ is a Sturm-Liouville operator.
  Then the Sturm-Liouville eigenvalue problem is 
  \[\hat{L}[y(x)] = -\lambda \omega(x)y(x)\]
  Where $\omega$ is a weight function and $x\in(a,b)$. The functions $p,p',q,\omega$ are
  continuous on $(a,b)$ and $p(x)>0,\omega(x)>0$ for all $x\in[a,b]$.
  Subject to unmixed boundary conditions, if $[a,b]$ is finite and the above assumptions
  hold on $[a,b]$ then the problem is \emph{regular} otherwise it \emph{singular}.
  \label{def:sturmLiouvilleOp}
\end{definition}
Several properties can be proven for the regular Sturm-Liouville eigenvalue problem. These
are a few of them:
\begin{enumerate}
  \item The eigenvalues are real, countable, ordered, and there is a smallest eigenvalue.
    We can write $\lambda_1<\lambda_2<\cdots$. There's no largest eigenvalue and it may be
    unbounded.
  \item For each eigenvalue $\lambda_n$ there's a corresponding eigenfunction $\phi_n$.
  \item Eigenfunctions corresponding to different eigenvalues are orthogonal (with respect
    to the weight function $\omega(x)$).
  \item The set of eigenfunctions is \emph{complete}, i.e. any piecewise smooth function
    can be represented by an eigenfunction expansion.
\end{enumerate}

