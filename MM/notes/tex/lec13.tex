\section{Lecture 13 - 18 Oct 2021}
Examples of exact equations, self-adjoint operators, 
\begin{example}
  We have 
  \[L[y]=(x^2-x)y'' + (2x^2+4x-3)y' + 8xy = 1\]
  Find the adjoint and use it to find the general solution.
\end{example}
\begin{proof}[Solution]
  Following the same proceadure as before, we multiply by $z$, and unknown, and use the
  condition $a''-b'+c =0$.We then get the adjoint equation as
  \[((x^2-x)z)'' - ((2x^2+4x-3)z)' + 8xz =0\]
  \[\implies (x^2-x)z'' - (2x^2-1)z' + (4x-2)z=0\]
  After re-arranging. And so,
  \[\overline{L}= (x^2-x)\frac{d^2}{dx^2} - (2x^2-1)\frac{d}{dx} + (4x-2)\]
  Recall that we want to rewrite the equation in an exact-form, so we need
  \[azy''+bzy'+czy = (Uy'+Vy)' = Uy''+(U'+V)y'+V'y\]
  Again, by equating coefficients we find
  \[U=(x^2-x)z\]
  \[U'+V = (2x^2+4x-3)z\]
  \[V'=8xz\]
  So by solving $\overline{L}[z]=0$ we find $z$ and hence find $U,V$ to get the exact
  equation. Recall that we only need one solution for $z$, not the general one. Try a
  solution of the form $z(x)=x^r$. We get
  \[(x^2-x)r(r-1)x^{r-2} - rx^{r-1}(2x^2-1) + x^r(4x-2)=0\]
  And rearranding coefficients we get
  \[2(2-r)x^{r+1}+(r-2)(r+1)x^r -r(r-2)x^{r-1}=0\]
  Since we know each function is linearly independent, we have that each coefficient must
  be $0$. $r=2$ satisfy the equation, hence the solution we're looking for is $z(x)=x^2$.
  We then can find $U,V$,
  \[U=z(x^2-x) = x^4-x^3\]
  \[V=-U'+z(2x^2+4x-3) = 2x^4\]
  And so our equation reduces to
  \[(Uy'+Vy)'= z\cdot 1\]
  \[\implies (x^4-x^3)y' +2x^4y = \int x^2 dx\]
  \[\implies (x^4-x^3)y' +2x^4y = \frac{x^3}{4}+c_1\]
  Using integrating factor $\mu=e^{2x}(x-1)^2$ we have 
  \[(y\cdot e^{2x}(x-1)^2)'= \frac{e^2x}{x^4-x^3} (\frac{x^{3}}{3} +c_1)(x-1)^2\]
  \[y e^{2x}(x-1)^{2} = \int\frac{1}{3}(x-1)e^{2x}dx + c_1\int \frac{e^2x(x-1)}{x^3}dx +
  c_2\]
  Hence by integrating and re-arranging we get the general solution.
  \[y= \frac{1}{(x-1)^2} (\frac{x}{6}-\frac{1}{4} + \frac{c_1}{x^2} +c_2e^{-2x})\]
\end{proof}


\subsection{Self adjoint operators}
\begin{definition}[Self adjoing operator]
  A linear operator $L$ is self adjoint if $L=\overline{L}$.
  \label{def:selfAdjOp}
\end{definition}
\begin{example}[Legendre's equation]
  Let $L=(1-x^2)\frac{d^2}{dx^2}- 2x\frac{d}{dx} + \lambda$.
  Show that $L$ is self-adjoint.
\end{example}
\begin{proof}
To get the adjoint, we multiply $L[y]$ by $z$ and get $zL[y]=(Uy'+Vy)'$ and find
\[\overline{L}[z]=[(1-x^2)z]'' - [-2xz]'+\lambda z\]
\[= (1-x^2)z'' + 2(-2x)z' + (-2)z + (2xz')+2z +\lambda z\]
\[=(1-x^2)z''-2xz'+\lambda z\]
\[=L[z]\]
Hence the operator is self adjoint.
\end{proof}

In fact, we can write any seconde order ODE in self adjoint forms. We can solve then the
equation in self-adjoint form and write the solution in canonical form.


\begin{theorem}
  A necessary condition for the operator 
  \[L=a\frac{d^2}{dx^2} +b\frac{d}{dx}+c\]
  To be self adjoint is that $a'=b$. Given this, the self adjoint operator can be written
  as 
  \[L=\frac{d}{dx}(a\frac{d}{dx})+c\]
  \label{<+label+>}
\end{theorem}
\begin{proof}
  We have $\overline{L}[y]= (ay)''-(by)'+cy = ay''+(2a'-b)y'+ (a''-b'+c)y$. Note that for
  $L=\overline{L}$ we require $2a'-b=b\implies a'=b$ and $a''-b'+c=c\implies a''=b'$. So
  $L$ is self adjoint if $a'=b$.

\end{proof}<++>
