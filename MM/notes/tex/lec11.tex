\section{Lecture 11 - 13 Oct 2021}
In the last lecture, we were in the middle of a problem with a repeated root. We found one
solution,
\[y_1 = \sum_{n=0}^{\infty} \frac{(-1)^m x^{2m}}{2^{2m}(m!)^2}\]
We can find the second solution $y_{2}$ by using Abel. However, this seems to be trickey,
since the first solution is a power series. That's why we will take a different approach
(in theory you can do that, but it's going to be rough).

Rewrite the expansion of $L[y]$ we have, (not knowing what $r$ is),
\[a_0 r^2 x^r + a_1 (r+1)^2x^{r+1} + \sum_{n=2}^{\infty} [a_n (n+r)^2 + a_{n-1}]\]
However, the second term is $0$ since we know $a_1=0$. Moreover, the addition is
$0$ given by the recurrence relation,
\[L[y] = a_0r^2x^r\]
We can compute the derivative wrt $r$ of the operator,
\[L[\frac{\partial y}{\partial r}] = a_0 2x x^r + a_0r^2 x^{r}\ln x\]
\[L[\frac{\partial y}{\partial r} (r=0)] = 0\]
And hence we find another solution: $y_2 = \frac{\partial y_1}{\partial r} (r=0)$. In this
scenario we find that the derivative wrt $r$ at $r=0$ is another solution, but in general,
for one-root solutions, the derivative at the root is a solution.
\[y_2 = \frac{\partial}{\partial r} \sum_{n=0}^{\partial} a_n(r) x^{n+r}\]
At $r=0$. So,
\[y_2 = x^r (\ln x \sum a_n(r) x^n + \sum \frac{\partial a_n}{\partial r} x^n)\]
\[y_2 = x^0 (\ln x \sum a_n(0) x^n) + \sum \frac{\partial a_n}{\partial r} x^n\]
\[y_2 = (\ln x)y_1 + \sum_{n=1}^{\infty} \frac{\partial a_n}{\partial r} x^n\]
Note we start adding at $n=1$ since $a_0$ is just a constant, so the derivative will be 0.
Let's look into the recurrence relation to find the derivative expression,
\[a_n = \frac{-a_{n-2}}{(r+n)^2}\]
\[\frac{\partial a_n}{\partial r} = \frac{2a_{n-2}}{(r+n)^3} - \frac{1}{(r+n)^2}
\frac{\partial a_{n-2}}{\partial r}\]
With a bit of calculation, we find without much trouble we find
\[a_{2}' = \frac{a_0}{4}\]
\[a_4' =\frac{-3a_0}{4^2 2^3}\]
We find the first couple of terms in $y_2$ as 
\[y_2(x) = \ln(x) y_1 (x) + [\frac{1}{4}x^2 - \frac{3}{128}x^4] + \cdots\]
For $x>0$. Note that $y_2$ has a logarithmic singularity at $x=0$. The general solution is
\[y = c_1 y_1 + c_2 y_2\]
This is the end of the chapter on power series.

\subsection{Chapter 3 - Inhomogeneous equations}
We will now look into equations of the form $L[y]=f(x)$. It turns out that given a
homogeneous solution, it's rather easy to find the solution to the inhomogenenous one.
\begin{theorem}
  Every solution $y$ of the inhomogeneous equation $L[y]=f(x)$ with
  \[L[y]=y'' + py' + qy\]
  Where $p,q,f$ are continuous functions, is of the form
  \[y = c_1y_1 + c_2y_2 + y_p\]
  Where $y_1,y_2$ are fundamental solutions to $L[y]=0$ and $y_p$ is \emph{any} solution
  to the inhomogeneous equation
\end{theorem}
\begin{proof}
  Let $y$ be s.t. $L[y]=f$, the general solution. We already have $y_p$ s.t. $L[y_p]=f$. Hence
  \[L[y]-L[y_p]=0\]
  \[\implies L[y-y_p]=0\]
  By linearity of $L$. We see then that $y-y_p$ is a solution to the homogeneous equation.
  All solutions to the homogeneous equation can be written as 
  \[y^*=c_1 y_1 + c_2 y_2 \]
  for some $c_1, c_2$, hence
  \[y-y_p = c_1 y_1 + c_2 y_2 \iff y = c_1 y_1 + c_2 y_2 + y_p\]
\end{proof}
To find $y_p$, we usually take an educated guess.

\subsection{Exact 2nd Order Equations}
The general equation has form
\[a y'' + by' + cy = f(x)\]
For functions of $x$, $a,b,c,f$. Which can always be rewritten as
\[(ay' -a'y + by)' + y(a''-b'+c) = f(x)\]
Checking this is left as an exercise for the reader. The differential equation is called
an \emph{exact equation} if 
\[a''(x) -b'(x) +c(x) =0\]
For all $x$ in the domain. Note that then we have
\[ay'+ (b-a')y = \int f(x) dx\]
This can be solved using an integrating factor.

