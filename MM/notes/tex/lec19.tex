\section{Lecture 19 - 1 Nov 2021}
\subsection{The weight function}
In the last example, where we were solving the eigenfunction problem
$L[y]=(xy')'+\frac{2}{x}y=-\lambda\omega(x)y$, the weight function caused a lot of
confusion. In our example, we were given an arbitrary $\omega$ function but we chose
$\omega=\frac{1}{x}$ to make the problem easier. However, one may think that choosing
$\frac{2}{x}$ would give a different result, but the Lecturer argues that what changes
is just a scaling of the eigenvalue.

The following is an attempt of the author to answer to the question where does the weight
funciton in the Sturm-Liouville eigenfunction problem come from. 
\todo{Go to office hours or consult somewhere else, check this is true.}
Starting from the definition, the eigenfunction problem is set as $L[y]=-\lambda y$
(see Definition \ref{def:eigenFun}). We have by Theorem \ref{thm:anyLinAdj} that any
operator of the form $ay''+by'+cy=-\lambda y$ for functions $a,b,c\in \SCC^2[a,b]$,
can be written in self-adjoint form (or Sturm-Liouville operator by Definition
\ref{def:sturmLiouvilleOp}),
\[(py')'+qy=H\]
With 
\[p(x)=\exp{\int\frac{b(x)}{a(x)}dx}\]
\[q(x)=p(x)\frac{c(x)}{a(x)}\]
\[H(x)=p(x)\frac{-\lambda y}{a(x)}\]
Which yields an eigenfunction problem of the form 
\[(py')'+qy=-\lambda \omega(x) y\]
With $\omega(x)=\frac{p(x)}{a(x)}$. This is where the weight function comes from. This
weight function will vary with different boundary conditions. 

\subsection{Eigenfunction expansion method}
To solve the BVP
\[L[y]=\frac{d}{dx}(p(x)\frac{d}{dx}y) + q(x) y = H(x)\]
Where $H\in\SCC[a,b]$ and unmixed boundary conditions. We start by finding the
eigenvalues $\lambda_1,\lambda_2,\cdots$ and the eigenfunctions $\phi_1,\phi_2,\cdots$ and
an appropriate weight function $\omega(x)$ so as to make the problem as simple as
possible. We then proceed to solve the problem
\[\hat{L}[\phi_n]=-\lambda \omega(x) \phi_n\]
Then assume that our solution $y(x)$ can be written as 
\[y(x)=\sum_{n=1}^{\infty} \gamma_n \phi_n(x)\]

Next, substitute this assumption into the equation we are interested in,
\[H(x)= \sum_{n=1}^{\infty} \gamma_n \hat{L}[\phi_n]\]
Since the operator is linear. Moreover, since $\hat{L}[\phi_n]=-\lambda_n\omega(x)\phi_n$,
\[H(x)= \sum_{n=1}^{\infty} -\gamma_n\lambda_n\omega(x)\phi_n(x)\]
Since the eigenfunctions are orthogonal, we can find $\gamma_n$. Multiply by $\phi_m$ for
$m\neq n$ and integrate over boundaries,
\[\int_a^b \phi_m(x)H(x)dx = -\sum_{n=1}^{\infty} \gamma_n\lambda_n
\int_a^b\omega(x)\phi_m\phi_n dx\]
Since $\int_a^b \omega(x)\phi_m\phi_n dx=0$ for $n\neq m$ we get 
\[\int_a^b \phi_m(x)H(x)dx = -\gamma_m\lambda_m
\int_a^b\omega(x)\phi_m\phi_m dx\]
We then solve for $\gamma_m$,
\[ \gamma_m = \frac{-\int_a^bH(x)\phi_m(x) dx}{\lambda_m\int_a^b\phi_m^2(x)\omega(x)dx}\]


\begin{example}
  Solve the BVP
  \[L[y]=(xy')' + \frac{2}{x}y = \frac{1}{x}\]
  At $x\in [1,2]$ and $y'(1)=0, y'(2)=0$.
\end{example}
\begin{proof}[Solution]
  The equation is already in self adjoint form. We have already solved the associated
  eigenvalue problem in the previous lecture. We chose the weight function
  $\omega=\frac{1}{x}$. The eigenvalues $\lambda_n=-2+\left( \frac{n\pi}{\log 2}
  \right)^2$ and the eigenfunctions
  \[\phi_{n}= A_n\cos \left( \frac{n\pi}{\log 2} \log x \right)\]
  It's useful to normalise the eigenfunctions so we choose $A_n$ s.t. the norm of $\phi_n$
  is $1$, i.e. 
  \[||\phi_n||=\int_1^2 \phi_n^2(x)\omega(x) dx = 1\]
  It can be solved by change of variables and trigonometric identities. The solution is,
  for $n\geq 1$,
  \[1=A_n^2 \frac{\log 2}{2} \implies A_n= \sqrt{\frac{2}{\log 2}}\]
  When $n=0$, we simply have 
  \[A_0= \sqrt{\frac{1}{\log 2}}\]
  Hence we have
  \[\phi_0=\frac{1}{\log 2}\]
  \[\phi_n = \frac{2}{\log 2} \cos\left( \frac{n\pi}{\log 2} \log x \right)\]
  We now solve the BVP. We do this by expanding $y$ in terms of the eigenfunctions.
  \[y(x) = \gamma_0 \frac{1}{\sqrt{\log 2}} + \sum_{n=1}^{\infty} \gamma_n
  \sqrt{\frac{2}{\log 2}} \cos\left( \frac{n\pi}{\log 2}\log x \right)\]
  Substituting back into the original equation we want to solve we find
  \[\frac{1}{x} = \sum_{n=0}^{\infty}\gamma_n L[\phi_n] = -
  \sum_{n=0}^{\infty}\gamma_n\lambda_n \phi_n(x)\omega(x) dx\]
  Using orthogonality as above we have
  \[\int_1^2 \frac{1}{x} \left( \sqrt{\frac{2}{\log 2}}\cos\left( \frac{m\pi}{\log 2}\log
  x\right) \right)dx  = -\lambda_m\gamma_m\]
  This integral evaluates to $0$ when $m\geq 1$. When $m=0$, we have 
  \[\gamma_0=\frac{\sqrt{\log 2}}{2}\]
  Since $\lambda_0=-2$.
  Hence 
  \[y(x)= \gamma_0 \frac{1}{\sqrt{\log 2}}= \frac{1}{2}\]
  The solution to the BVP.
\end{proof}
