\section{Lecture 12 - 15 Oct 2021}
\begin{definition} [Exact equation]
  Let us consider the equation 
  \[a y'' + by' + cy = f(x)\]
  Then we have an \emph{exact equation} if 
  \[a''-b'+c =0\]
  Hence we have
  \[(ay'-a'y +by)' = f(x)\]
  \label{exactEq}
\end{definition}
\begin{example}
  Find the general solution to 
  \[\frac{1}{x}y'' + (\frac{1}{x} - \frac{2}{x^2})y' - (\frac{1}{x^2} -
  \frac{2}{x^3})y = e^x\]
\end{example}
\begin{proof}[Solution]
  We have,
  \[a' = \frac{-1}{x^2}\]
  \[a'' = \frac{2}{x^3}\]
  \[b' = \frac{-1}{x^2} + \frac{4}{x^3}\]
  And so,
  \[a''-b'+c = \frac{2}{x^3} - (\frac{-1}{x^2} - \frac{4}{x^3}) - (\frac{1}{x^2} -
  \frac{2}{x^3}) =0\]
  And so our equation is exact. Hence we have
  \[\frac{1}{x}y' + \frac{1}{x^2}y + (\frac{1}{x}-\frac{2}{x^2})y = \int e^x dx\]
  \[\frac{1}{x}y' + y(- \frac{1}{x^2} + \frac{1}{x}) = e^x +c_1 \]
  \[y' + y(- \frac{1}{x} + 1) = xe^x +xc_1 \]
  Hence we can use an integrating factor
  \[\mu (x) = \exp \left\{ \int 1 -\frac{1}{x} dx \right\}\]
  \[\mu (x) = \frac{e^x}{x}\]
  Hence we get
  \[(y \frac{e^x}{x})' = \frac{e^x}{x}(xe^x +xc_1)\]
  \[y = xe^{-x} \int e^{2x} + c_1e^x dx\]
  \[y = \frac{x}{2} e^x + c_1 x + c_2 xe^{-x}\]
  For some constants $c_1,c_2$.
\end{proof}

\subsection{Adjoint equations and integrating factors}
If the equations can be multiplied by a function $z(x)$ so that the resulting equation is
exact, then $z(x)$, then it's called \emph{integrating factor} for that equation.
Consider,
\[ay''+by'+cy=f(x)\]
Then with our integrating factor,
\[zay'' + zby' + czy = zf\]
The equation is exact if we can write this as 
\[zay'' + zby' + czy =\frac{d}{dx}(U(x)y' + V(x) y)\]
If such a $z$ exists, we have
\[\frac{d}{dx}(U(x)y' + V(x) y) = zf\]
\[\implies Uy' + Vy = \int zf dx\]
We want to calculate $z,U,V$. We have 
\[zay'' + zby' + czy = U y'' + (U'+V)y' + V' y\]
We have
\[U = za\]
\[(U'+V) = zb\]
\[V' = zc\]
Differentiating the first and second equations,
\[U'' + V' = z'b + b'z\]
\[U' = z'a + a'z\]
And subtracting the fifth one by the second one we get 
\[V = z(b-a') -z'b\]
We then can differentiate and get two expressions for $V'$,
\[V' = (zb)'-(za)''\]
And so, 
\[(za)'' -(zb)' + zc = 0\]
This is called \emph{adjoint equation}. Let us define
\[\bar{L} [z] = \frac{d^{2}}{dx^2}(za) - \frac{d}{dx}(zb) + zc\]
Where the operator $\bar{L}[z]$ is called the adjoint operator of $L[y]$.
This equation is easier to solve than the original equation. When solved, we can get
$U,V$, which depend on $z$ and things we know.

\begin{example}
  Solve the following equation with the adjoint,
  \[L[y] = y''+4y=x^2\]
\end{example}
\begin{proof}[Solution]
  We have
  \[zy'' +4zy= (Uy' + Vy)'\]
  So equating coefficients,
  \[z=U\]
  \[0=U'+V \implies V=-U'=-z'\]
  \[4z = V' \implies 4z= -z''\]
  We also have
  \[(za)'' - (zb)' + (zc)' = z'' - 0 +4z =0\]
  \[z'' = -4z\]
  \[\bar{L}[z]=0\]
  And solving this homogeneous equation gives $z$. We use a solution of the form
  $z=x^{mx}$ and the auxiliar equation gives
  \[m^2 + 4 = 0\]
  \[\implies m= \pm 2 i\]
  Since $z$ is only an integrating factor, we only need one solution, and not a general
  one. We use the solution $2z=e^{2ix}+e^{-2ix} = 2\sin(2x)$.
  From here,
  \[U = z = \sin(2x)\]
  \[V = -U' = -2\cos(2x)\]
  And so our original equation becomes
  \[\frac{d}{dx} (\sin(2x) \frac{dy}{dx} - 2\cos(2x) y) = x^2 \sin(2x)\]
  So integrating, 
  \[\sin(2x)y' -2\cos(2x)y = \int x^2 \sin(2x) dx\]
  \[\implies y' -2\cot(2x)y = \sin^{-1}(2x)\int x^2 \sin(2x) dx\]
  Then using the integrating factor $\mu = \frac{1}{\sin 2x}$ we get 
  \[y = \sin(2x) \int \frac{1}{sin^2(2x)}\int s^2 \sin (2s) ds dx\]
  \[y = c_1 \sin(2x) + c_2 \cos(2x) - \frac{1}{8} + \frac{1}{4}x^2\]
  The computation is left as an exercise to the reader. The last part is the particular
  solution, and the whole $y$ is the general solution.

\end{proof}
