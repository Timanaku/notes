\section{Lecture 24 - 12 Nov 2021}
\subsection{Another example of Green's functions applied to IVPs}
\begin{example}
  Solve the IVP $y''+y=f(t)$, with $y(0)=y'(0)=0$.
\end{example}
\begin{proof}[Solution]
  Considering the Green's function$G(t;\tau)$, we have $L[G]=0$ for $0\leq t<\tau$, so
  $G(t,\tau)=A(\tau)\cos(t)+B(t)\sin(t)$, with ICs $A(\tau)=B(\tau)=0$, hence $G(t)=0$.

  On the other hand, for $\tau<t<\infty$, the same yields
  $G(t;\tau)=C(\tau)\cos(t)+D(\tau)\sin(t)$, where continuity and jump conditions yield
  \[C(\tau)\cos(\tau)+D(\tau)\sin(\tau)=0\]
  \[-C(\tau)\sin(\tau)+D(\tau)\cos(\tau)=1\]
  Putting it in matrix form we find rather easily that $C(\tau)=-\sin(\tau)$, and
  $D(\tau)=\cos(\tau)$. Therefore we have 
  \[ G(t;\tau) = 
    \begin{cases}
      0, & 0\leq t< \tau\\
      -\sin(\tau)\cos(t) + \cos(\tau)\sin(t), & \tau< t< \infty
    \end{cases}
  \]
  \[ G(t;\tau) = 
    \begin{cases}
      0, & 0\leq t< \tau\\
      \sin(t-\tau), & \tau< t< \infty
    \end{cases}
  \]
  So the most general solution is
  \[y(t)=\int_0^t \sin(t-\tau)f(\tau) d\tau\]
\end{proof}
\subsection{First order PDE}
ODEs contain 1 independent variable, whereas PDEs contain 2 or more independent variables.
We will consider equations with dependent variable $u$ and independent variables $x,y$
with partial derivatives $u_x,u_y$. 
\begin{definition}
  Let $x,y\in\RR$ be independent variables. Let $u\in\RR$ be a dependent variable on
  $x,y$. A $n$th order PDE is an expresion of the form 
  \[F(x,y,u,u_x,u_y, u_{xx},u_{xy},u_{yx}, u_{yy}, \cdots)=0\]
  Where $u_{s_1s_2s_3\cdots s_n}$ (where $s_i\in \{x,y\}$) is the $n$th partial derivative
  wrt $s_1,s_2,\cdots,s_n$.
  \label{def:firstOrdPDE}
\end{definition}
In this section we will only care about first-order PDE. That is, expressions of the form
$F(x,y,u,u_x,u_y)=0$.
\subsection{Constant coefficient linear PDEs}
Consider the \emph{constant coefficient PDE} $au_x +bu_y=0$ with $a^2+b^2>0$. Note that
this is exactly the vector equation $\nabla u.b=0$ where $.$ is the vector product,
$\nabla u=(u_x,u_y)$, and $b=(a,b)$ with $|b|>0$. This equation is really saying that the
\emph{directional derivative} of $u$ in the direction $b$ is 0 -- that $u$ is constant in
the direction of $b$. Lines in the direction of $b$ have the form $bx-ay=c$, for
$c\in\RR$.
\begin{definition}[Characteristic Lines]
  Given a constant coefficients linear PDE of the form $\nabla u\cdot b=0$ for $\nabla
  u\in\RR^2$ and $b=(a,b)\in\RR^2$, characteristic lines are lines in the direction of
  $b$, i.e. lines $bx-ay=c$ for any $c\in\RR$. Along these lines, $u$ is constant.
  \label{def:charLines}
\end{definition}
Following the above problem, we have that $u(x,y)=f(c)$ is the solution, since $u$ is
constant along lines $c=bx-ay$. This can easily be shown.
