\section{Lecture 26 - 17 Nov 2021}
\begin{example}
  Solve $u_x-u_y+u+x-y+2=0$.
\end{example}
\begin{proof}[Solution]
  As seen in the Definition \ref{def:methodOfCharaceristics} we need the specification
  $a(x,y)\eta_x+b(x,y)\eta_y=0$, giving the characteristic equation $\frac{dy}{dx}=-1$,
  which gives the characteristic curves $y=-x+C$ for some $C\in \RR$. That is, along these
  characteristic curves, $\eta$ is constant: $\eta(x,y)=f(c)=f(x+y)$, for some
  differentiable function $f$. Since we can choose what $f$ suits us best, we choose the
  simples example, $\eta(x,y)=x+y$. As previously stated, we usually choose $\xi=x$ for
  simplicity, which leads to the Jacobian $J=1$, and hence our invertible change of
  variables are
  \[\eta=x+y \implies y=\eta-\xi\]
  \[\xi=x \implies x=\xi\]
  By rewriting the PDE in terms of the new coordinates, we get
  \[u_{\xi}+u+2\xi-\eta+2 =0\]
  Which is an ODE in $\xi$. After using integrating factors we arrive to the solution
  \[u(\xi,\eta)=-2\xi+\eta+A(\eta)e^{-
  \xi}\]
  For some function $A(\eta)$. Transforming back to the original coordinates, we get
  \[u(x,y)=(y-x)+A(x+y)e^{-x}\]
\end{proof}

\subsection{Particular solutions}
To find a particular solution with the above method, as in any method, we need to specify
boundary conditions or initial conditions. These conditions are encapsulated in the term
\emph{auxiliary conditions}. These auxiliary conditions are specified by a curve $\Gamma$,
a curve that lies on the solution surface (for a $u(x,y)$, think about $\RR^3$, where the
  xy-plane are the independent variables, and $u$ is the z-plane, specifying a scalar
field -- where along the characteristic curves, the field is constant). 
\begin{example}
  Find the general solution of $xu_x+yu_y-u=0$ for $x>0$, and find the particular solution
  subject to the condition $x=\cos\tau$, $y=-\sin\tau$, $u=1$.
\end{example}
\begin{proof}[Solution]
  Again, we need new coordinates where $a\eta_x+b\eta_y=0$, where $a=x,b=y$.
\end{proof}<++>
