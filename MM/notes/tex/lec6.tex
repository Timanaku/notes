\section{Lecture 6 - 1 Oct 2021}
From now on, we'll focus on Power series solutions to ODEs. Previously we'd find a
solution of a homogeneous ODE when we knew another solution (using Abel). We need some way
of finding the initial solution. Sometimes it's okay to find by inspections, but it's
usual that this is not effective. 

For a homogeneous ODE $a(x)y''+b(x)y'+c(x)y=0$, if $a,b,c$ are polynomials in $x$, then we
can look for power-series solutions. I.e. we look for solutions of the form
\[y(x)=\sum_{n=0}^{\infty} a_n(x-x_0)^n\]
We need to determine the coefficients $a_0,a_1,a_2,\cdots$.
\subsection{Review of Taylor series}
Recall that for a function $f(x)=\sum_{n=0}^{\infty} a_n(x-x_0)^n$, we have 
\[f'(x)= a_1 + 2a_2(x-x_0) + 3a_3(x-x_0)^2+\cdots\]
\[f''(x)=2a_2 + 3*2 a_3(x-x_0) + \cdots\]
Hence we have $f(x_0)=a_0, f'(x_0)=a_1, f''(x_0)=2a_2$, etc. We observe a pattern to find
the desired coefficients, called the Taylor expansion,
\[f(x)= f(x_0) + f'(x_0)(x-x_0) + f''(x_0)(x-x_0)^2/2! + \cdots\]

However, note that for the series to be useful, it must converge (otherwise, it's not
defined!). Recall that we have absolute convergence if 
\[\lim_{N\to \infty} \sum_{n=0}^N |a_n(x-x_0)^n|\]
Converges. By using the ratio test, we can check convergence,
\[\lim_{n\to \infty} |\frac{a_{n+1}(x-x_0)^{n+1}}{a_n(x-x_0)^n}|\]
\[=|x-x_0|\lim_{n\to\infty} |\frac{a_{n+1}}{a_n}=L\]
If $L<1$, the series converges. If $L=1$, the ratio test is inconclusive. If $L>1$, the
series diverges. 

Recall, the radius of convergence is the largest $x$ for which $L<1$. So if $L<1$ for some
$|x-x_0|<p$, this gives the radius of convergence.

\subsection{ Chapter 2 - Power series solutions}
Consider the ODE
\[a(x)y''+b(x)y'+c(x)y=0\]
With $a,b,c$ polynomials in $x$. We look for a power series expansion about $x_0$
\begin{definition}
  A point $x_o\in\RR$ is called ordinary of the equation above if $a(x_0)\neq 0$.
  Otherwise, the point is called singular.
  \label{ordinaryPoint}
\end{definition}

\begin{example}
  Solve $y''-xy'-y=0$ about $x_0=0$.
\end{example}
\begin{proof}[Solution]
  Every point is ordinary. THen, we have $y=\sum a_n x^n$, $y'=\sum na_nx^{n-1}$, and
  $y''=n(n-1)a_nx^{n-2}$. Substituting in the equation at the tob, we have (all the sums
  being from 0 to infinity).
  \[ \sum n(n-1)a_nx^{n-2} - x\sum na_nx^{n-1} - \sum a_n x^n = 0\] 
  \[\iff \sum (n (n-1)a_n x^{n-2} - na_nx^{n} - a_n x^n ) = 0\]
  Note that $x^0,x,x^2,\cdots$ are all linearly independent, so every element should
  vanish. Hence, every coefficient should be 0. Note that we can rewrite the above as 
  \[\sum ((N+2)(N+1)a_{N+2} x^N - Na_Nx^N - a_Nx^N)=0\]
  \[\iff\sum ((N+2)(N+1)a_{N+2}  - Na_N - a_N)x^N=0\]
  Hence we must have that for every $N$, 
  \[(N+2)(N+1)a_{N+2} - (N+1)a_N = 0\]
  \[\iff a_{N+2}= \frac{a_N}{N+2}\]
  This is a recurrence relation, which will be useful in the future. Given $a_0$ we can
  find out even coefficients, and given $a_1$ we can find out odd coefficients.
  Hence we can write
  \[y = a_{0} (1+ x^{2}/2 + x^{3}/2*4 + x^{4}/2*4*6+\cdots) + a_1(x + x^3/3 + x^{5}/3*5 +
  \cdots) \]
  \[y= a_{0}\sum \frac{x^2n}{2^n n!} + a_1 \sum \frac{2^n n! x^{2n+1}}{(2n+1)!}\]
  And this is exactly what we'd expect to see, a linear combination of two linearly
  independent functions, the series.


\end{proof}
